\subsection{DHT22}
\begin{lstlisting}[style=mystyle,caption={DHT22code}]
#!/home/mistaherd/Documents/Github/meshnetwork_in_forest/env/lib/python3.11
import adafruit_dht 
import board
import datetime
import pandas as pd
class DHT22:
##Set DATA pin to pin 4
    def __init__(self):
        # self.dhtDevice =adafruit_dht.DHT22(board.D4)
        self.dhtDevice =adafruit_dht.DHT11(board.D4)
    def Read_DHT22_data(self)-> tuple[float,float,str]:
        try:
            Humidity=self.dhtDevice.humidity
            Temperature=self.dhtDevice.temperature
            timestamp =datetime.datetime.now()
            timestamp = timestamp.strftime("%Y-%m-%d %H:%M:%S")
            return Temperature,Humidity,timestamp
        except RuntimeError as e:
            print(f"Error reading sensor: {e}")
            return None, None
    def write_to_csv(self,filename:str):
        temperature, humidity, timestamp = self.Read_DHT22_data()
        if temperature is not None and humidity is not None and timestamp is not None:
            data = [(temperature, humidity, timestamp)]
            df = pd.DataFrame(data, columns=['Temperature', 'Humidity', 'Timestamp'])
            df.to_csv(filename, index=False)
        else:
            print("Failed to retrieve data from sensor. Data not written to CSV.")
dht_sensor = DHT22()
dht_sensor.write_to_csv("sensor_data.csv")
\end{lstlisting}
\newpage
\subsection{AS312}
\begin{lstlisting}[style=mystyle,caption={code for  AS312}]
#!/home/mistaherd/Documents/Github/meshnetwork_in_forest/env/lib/python3.11
import RPi.GPIO as GPIO
import time
import datetime
import pandas as pd
#pin 17
class AS312:
	def __init__(self,pin_number:int):
		self.pin_number=pin_number
		self.GPIO=GPIO
		self.GPIO.setmode(GPIO.BCM)
		self.GPIO.setup(self.pin_number,GPIO.IN)
		self.current_state=0
		self.timestamp=datetime.datetime.now().strftime("%Y-%m-%d %H:%M:%S")
	def read_state(self)->int:
		self.current_state =self.GPIO.input(self.pin_number)
		return self.current_state
	def append_data(self):
		data={
			"Motion Dectected": [self.current_state],
			"Timestamp": [self.timestamp]
		}
		df =pd.DataFrame(data)
		df.to_csv('sensor_data.csv',mode='a' ,index=False,header=False)
pir_sensor = AS312(17)
try:
	time.sleep(0.1)
	current_state =pir_sensor.read_state()
	timestamp=pir_sensor.timestamp
	print("GPIO pin %s is %s" % (pir_sensor.pin_number,current_state))
	if current_state == 1:
		print("Motion dectected")
	pir_sensor.append_data()
except KeyboardInterrupt:
	pass
finally:
	GPIO.cleanup()
\end{lstlisting}
\newpage
\subsection{ADC}
\begin{lstlisting}[style=mystyle,caption={Code for ADC }]
    #!/home/mistaherd/Documents/Github/meshnetwork_in_forest/env/lib/python3.11
'''!
  @file DFRobot_ADS1115.py
  @brief Provides an Raspberry pi library to read ADS1115 data over I2C. Use this library to read analog voltage values.
  @copyright   Copyright (c) 2010 DFRobot Co.Ltd (http://www.dfrobot.com)
  @license     The MIT License (MIT)
  @author [luoyufeng](yufeng.luo@dfrobot.com)
  @version  V1.0
  @date  2019-06-19
  @url https://github.com/DFRobot/DFRobot_ADS1115
'''

import smbus
import time

## Get I2C bus
bus = smbus.SMBus(1)

## I2C address of the device
ADS1115_IIC_ADDRESS0				= 0x48
ADS1115_IIC_ADDRESS1				= 0x49

## ADS1115 Register Map
## Conversion register
ADS1115_REG_POINTER_CONVERT			= 0x00 
## Configuration register
ADS1115_REG_POINTER_CONFIG			= 0x01 
## Lo_thresh register
ADS1115_REG_POINTER_LOWTHRESH		= 0x02 
## Hi_thresh register
ADS1115_REG_POINTER_HITHRESH		= 0x03 

## ADS1115 Configuration Register
## No effect
ADS1115_REG_CONFIG_OS_NOEFFECT		= 0x00 
## Begin a single conversion
ADS1115_REG_CONFIG_OS_SINGLE		= 0x80 
## Differential P = AIN0, N = AIN1 (default)
ADS1115_REG_CONFIG_MUX_DIFF_0_1		= 0x00 
## Differential P = AIN0, N = AIN3
ADS1115_REG_CONFIG_MUX_DIFF_0_3		= 0x10 
## Differential P = AIN1, N = AIN3
ADS1115_REG_CONFIG_MUX_DIFF_1_3		= 0x20 
## Differential P = AIN2, N = AIN3
ADS1115_REG_CONFIG_MUX_DIFF_2_3		= 0x30 
## Single-ended P = AIN0, N = GND
ADS1115_REG_CONFIG_MUX_SINGLE_0		= 0x40 
## Single-ended P = AIN1, N = GND
ADS1115_REG_CONFIG_MUX_SINGLE_1		= 0x50 
## Single-ended P = AIN2, N = GND
ADS1115_REG_CONFIG_MUX_SINGLE_2		= 0x60 
## Single-ended P = AIN3, N = GND
ADS1115_REG_CONFIG_MUX_SINGLE_3		= 0x70 
## +/-6.144V range = Gain 2/3
ADS1115_REG_CONFIG_PGA_6_144V		= 0x00 
## +/-4.096V range = Gain 1
ADS1115_REG_CONFIG_PGA_4_096V		= 0x02 
## +/-2.048V range = Gain 2 (default)
ADS1115_REG_CONFIG_PGA_2_048V		= 0x04 
## +/-1.024V range = Gain 4
ADS1115_REG_CONFIG_PGA_1_024V		= 0x06 
## +/-0.512V range = Gain 8
ADS1115_REG_CONFIG_PGA_0_512V		= 0x08 
## +/-0.256V range = Gain 16
ADS1115_REG_CONFIG_PGA_0_256V		= 0x0A
## Continuous conversion mode
ADS1115_REG_CONFIG_MODE_CONTIN		= 0x00 
## Power-down single-shot mode (default)
ADS1115_REG_CONFIG_MODE_SINGLE		= 0x01 
## 8 samples per second
ADS1115_REG_CONFIG_DR_8SPS			= 0x00 
## 16 samples per second
ADS1115_REG_CONFIG_DR_16SPS			= 0x20 
## 32 samples per second
ADS1115_REG_CONFIG_DR_32SPS			= 0x40 
## 64 samples per second
ADS1115_REG_CONFIG_DR_64SPS			= 0x60 
## 128 samples per second (default)
ADS1115_REG_CONFIG_DR_128SPS		= 0x80
## 250 samples per second
ADS1115_REG_CONFIG_DR_250SPS		= 0xA0 
## 475 samples per second
ADS1115_REG_CONFIG_DR_475SPS		= 0xC0 
## 860 samples per second
ADS1115_REG_CONFIG_DR_860SPS		= 0xE0 
## Traditional comparator with hysteresis (default)
ADS1115_REG_CONFIG_CMODE_TRAD		= 0x00 
## Window comparator
ADS1115_REG_CONFIG_CMODE_WINDOW		= 0x10 
## ALERT/RDY pin is low when active (default)
ADS1115_REG_CONFIG_CPOL_ACTVLOW		= 0x00 
## ALERT/RDY pin is high when active
ADS1115_REG_CONFIG_CPOL_ACTVHI		= 0x08 
## Non-latching comparator (default)
ADS1115_REG_CONFIG_CLAT_NONLAT		= 0x00 
## Latching comparator
ADS1115_REG_CONFIG_CLAT_LATCH		= 0x04 
## Assert ALERT/RDY after one conversions
ADS1115_REG_CONFIG_CQUE_1CONV		= 0x00 
## Assert ALERT/RDY after two conversions
ADS1115_REG_CONFIG_CQUE_2CONV		= 0x01 
## Assert ALERT/RDY after four conversions
ADS1115_REG_CONFIG_CQUE_4CONV		= 0x02 
## Disable the comparator and put ALERT/RDY in high state (default)
ADS1115_REG_CONFIG_CQUE_NONE		= 0x03 

mygain=0x02
coefficient=0.125
addr_G=ADS1115_IIC_ADDRESS0
class ADS1115():
	def set_gain(self,gain):
		'''!
		  @brief Sets the gain and input voltage range.
		  @param gain  This configures the programmable gain amplifier
		  @n ADS1115_REG_CONFIG_PGA_6_144V     = 0x00 # 6.144V range = Gain 2/3
		  @n ADS1115_REG_CONFIG_PGA_4_096V     = 0x02 # 4.096V range = Gain 1
		  @n ADS1115_REG_CONFIG_PGA_2_048V     = 0x04 # 2.048V range = Gain 2
		  @n default:
		  @n ADS1115_REG_CONFIG_PGA_1_024V     = 0x06 # 1.024V range = Gain 4
		  @n ADS1115_REG_CONFIG_PGA_0_512V     = 0x08 # 0.512V range = Gain 8
		  @n ADS1115_REG_CONFIG_PGA_0_256V     = 0x0A # 0.256V range = Gain 16
		'''
		global mygain
		global coefficient
		mygain=gain
		if mygain == ADS1115_REG_CONFIG_PGA_6_144V:
			coefficient = 0.1875
		elif mygain == ADS1115_REG_CONFIG_PGA_4_096V:
			coefficient = 0.125
		elif mygain == ADS1115_REG_CONFIG_PGA_2_048V:
			coefficient = 0.0625
		elif mygain == ADS1115_REG_CONFIG_PGA_1_024V:
			coefficient = 0.03125
		elif mygain == ADS1115_REG_CONFIG_PGA_0_512V:
			coefficient = 0.015625
		elif  mygain == ADS1115_REG_CONFIG_PGA_0_256V:
			coefficient = 0.0078125
		else:
			coefficient = 0.125
	def set_addr_ADS1115(self,addr):
		'''!
		  @brief Sets the IIC address.
		  @param addr  7 bits I2C address, the range is 1~127.
		'''
		global addr_G
		addr_G=addr
	def set_channel(self,channel):
		'''!
		  @brief Select the Channel user want to use from 0-3.
		  @param channel  the Channel: 0-3
		  @n For Single-ended Output: 
		  @n    0 : AINP = AIN0 and AINN = GND
		  @n    1 : AINP = AIN1 and AINN = GND
		  @n    2 : AINP = AIN2 and AINN = GND
		  @n    3 : AINP = AIN3 and AINN = GND
		  @n For Differential Output:
		  @n    0 : AINP = AIN0 and AINN = AIN1
		  @n    1 : AINP = AIN0 and AINN = AIN3
		  @n    2 : AINP = AIN1 and AINN = AIN3
		  @n    3 : AINP = AIN2 and AINN = AIN3
		  @return channel
		'''
		global mygain
		self.channel = channel
		while self.channel > 3 :
			self.channel = 0
		
		return self.channel
	
	def set_single(self):
		'''!
		  @brief Configuration using a single read.
		'''
		global addr_G
		if self.channel == 0:
			CONFIG_REG = [ADS1115_REG_CONFIG_OS_SINGLE | ADS1115_REG_CONFIG_MUX_SINGLE_0 | mygain | ADS1115_REG_CONFIG_MODE_CONTIN, ADS1115_REG_CONFIG_DR_128SPS | ADS1115_REG_CONFIG_CQUE_NONE]
		elif self.channel == 1:
			CONFIG_REG = [ADS1115_REG_CONFIG_OS_SINGLE | ADS1115_REG_CONFIG_MUX_SINGLE_1 | mygain | ADS1115_REG_CONFIG_MODE_CONTIN, ADS1115_REG_CONFIG_DR_128SPS | ADS1115_REG_CONFIG_CQUE_NONE]
		elif self.channel == 2:
			CONFIG_REG = [ADS1115_REG_CONFIG_OS_SINGLE | ADS1115_REG_CONFIG_MUX_SINGLE_2 | mygain | ADS1115_REG_CONFIG_MODE_CONTIN, ADS1115_REG_CONFIG_DR_128SPS | ADS1115_REG_CONFIG_CQUE_NONE]
		elif self.channel == 3:
			CONFIG_REG = [ADS1115_REG_CONFIG_OS_SINGLE | ADS1115_REG_CONFIG_MUX_SINGLE_3 | mygain | ADS1115_REG_CONFIG_MODE_CONTIN, ADS1115_REG_CONFIG_DR_128SPS | ADS1115_REG_CONFIG_CQUE_NONE]
		
		bus.write_i2c_block_data(addr_G, ADS1115_REG_POINTER_CONFIG, CONFIG_REG)
	
	def set_differential(self):
		'''!
		  @brief Configure as comparator output.
		'''
		global addr_G
		if self.channel == 0:
			CONFIG_REG = [ADS1115_REG_CONFIG_OS_SINGLE | ADS1115_REG_CONFIG_MUX_DIFF_0_1 | mygain | ADS1115_REG_CONFIG_MODE_CONTIN, ADS1115_REG_CONFIG_DR_128SPS | ADS1115_REG_CONFIG_CQUE_NONE]
		elif self.channel == 1:
			CONFIG_REG = [ADS1115_REG_CONFIG_OS_SINGLE | ADS1115_REG_CONFIG_MUX_DIFF_0_3 | mygain | ADS1115_REG_CONFIG_MODE_CONTIN, ADS1115_REG_CONFIG_DR_128SPS | ADS1115_REG_CONFIG_CQUE_NONE]
		elif self.channel == 2:
			CONFIG_REG = [ADS1115_REG_CONFIG_OS_SINGLE | ADS1115_REG_CONFIG_MUX_DIFF_1_3 | mygain | ADS1115_REG_CONFIG_MODE_CONTIN, ADS1115_REG_CONFIG_DR_128SPS | ADS1115_REG_CONFIG_CQUE_NONE]
		elif self.channel == 3:
			CONFIG_REG = [ADS1115_REG_CONFIG_OS_SINGLE | ADS1115_REG_CONFIG_MUX_DIFF_2_3 | mygain | ADS1115_REG_CONFIG_MODE_CONTIN, ADS1115_REG_CONFIG_DR_128SPS | ADS1115_REG_CONFIG_CQUE_NONE]
		
		bus.write_i2c_block_data(addr_G, ADS1115_REG_POINTER_CONFIG, CONFIG_REG)
	
	def read_value(self):
		'''!
		  @brief  Read ADC value.
		  @return raw  adc
		'''
		global coefficient
		global addr_G
		data = bus.read_i2c_block_data(addr_G, ADS1115_REG_POINTER_CONVERT, 2)
		
		# Convert the data
		raw_adc = data[0] * 256 + data[1]
		
		if raw_adc > 32767:
			raw_adc -= 65535
		raw_adc = int(float(raw_adc)*coefficient)
		return {'r' : raw_adc}

	def read_voltage(self,channel):
		'''!
		  @brief Reads the voltage of the specified channel.
		  @param channel  the Channel: 0-3
		  @n For Single-ended Output: 
		  @n    0 : AINP = AIN0 and AINN = GND
		  @n    1 : AINP = AIN1 and AINN = GND
		  @n    2 : AINP = AIN2 and AINN = GND
		  @n    3 : AINP = AIN3 and AINN = GND
		  @n For Differential Output:
		  @n    0 : AINP = AIN0 and AINN = AIN1
		  @n    1 : AINP = AIN0 and AINN = AIN3
		  @n    2 : AINP = AIN1 and AINN = AIN3
		  @n    3 : AINP = AIN2 and AINN = AIN3
		  @return Voltage
		'''
		self.set_channel(channel)
		self.set_single()
		time.sleep(0.1)
		return self.read_value()

	def comparator_voltage(self,channel):
		'''!
		  @brief Sets up the comparator causing the ALERT/RDY pin to assert .
		  @param channel  the Channel: 0-3
		  @n For Single-ended Output: 
		  @n    0 : AINP = AIN0 and AINN = GND
		  @n    1 : AINP = AIN1 and AINN = GND
		  @n    2 : AINP = AIN2 and AINN = GND
		  @n    3 : AINP = AIN3 and AINN = GND
		  @n For Differential Output:
		  @n    0 : AINP = AIN0 and AINN = AIN1
		  @n    1 : AINP = AIN0 and AINN = AIN3
		  @n    2 : AINP = AIN1 and AINN = AIN3
		  @n    3 : AINP = AIN2 and AINN = AIN3
		  @return Voltage
		'''
		self.set_channel(channel)
		self.set_differential()
		time.sleep(0.1)
		return self.read_value()
\end{lstlisting}
\newpage
\subsection{DFR0026}
\begin{lstlisting}[style=mystyle,caption={Code for  DFR00026}]
#!/home/mistaherd/Documents/Github/meshnetwork_in_forest/env/lib/python3.11
from DFRobot_ADS1115 import ADS1115
import time
class DFR0026():
    def __init__(self):
        self.ADS1115_REG_CONFIG_PGA_6_144V        = 0x00 # 6.144V range = Gain 2/3
        self.ADS1115_REG_CONFIG_PGA_4_096V        = 0x02 # 4.096V range = Gain 1
        self.ADS1115_REG_CONFIG_PGA_2_048V        = 0x04 # 2.048V range = Gain 2 (default)
        self.ADS1115_REG_CONFIG_PGA_1_024V        = 0x06 # 1.024V range = Gain 4
        self.ADS1115_REG_CONFIG_PGA_0_512V        = 0x08 # 0.512V range = Gain 8
        self.ADS1115_REG_CONFIG_PGA_0_256V        = 0x0A # 0.256V range = Gain 16
        self.ads1115 = ADS1115()
        self.ads1115.set_addr_ADS1115(0x48)
        self.ads1115.set_gain(self.ADS1115_REG_CONFIG_PGA_6_144V)
        self.adc_channel=0
    def read_voltage(self):
        return self.ads1115.read_voltage(self.adc_channel)
        #time.sleep(0.2) after read it
light_vaule=DFR0026()
print(light_vaule.read_voltage())  

\end{lstlisting}
\newpage
\subsection{Camera}
\begin{lstlisting}[style=mystyle,caption={Code for Camera}]
#!/home/mistaherd/Documents/Github/meshnetwork_in_forest/env/lib/python3.11
from picamera import PiCamera
from time import sleep
from datetime import datetime
class Raspberry_Pi_VR_220:
    def __init__(self):
        """setup an instan  for the  camera"""
        self.timestamp=datetime.now().strftime("%Y-%m-%d_%H:%M:%S")
        self.fname ='/home/mistaherd/Documents/Github/meshnetwork_in_forest/{}.png'.format(self.timestamp)
        self.camera=PiCamera()
        self.timeamount=2
    def take_pic(self)-> str:
        """this will take  a picture from camera"""
        self.camera.start_preview()
        sleep(self.timeamount)
        self.camera.capture(self.fname)
        self.stop_preview()
        return self.fname
camera=Raspberry_Pi_VR_220()
picture=camera.take_pic()
\end{lstlisting}
\newpage 
\subsection{Memory mangement}
\begin{lstlisting}[style=mystyle,caption={Code for  memory mangement }]
#!/home/mistaherd/Documents/Github/meshnetwork_in_forest/env/lib/python3.11
import pandas as pd
# from DHT22 import DHT22

# from AS312 import AS312
# from MCP3008 import DF0026
import pandas as pd
import glob
import re 
import subprocess
class sensor_data:
	def __init__(self):
		self.dht22 = DHT22()
		self.humidity,self.temperature,self.timestamp=self.dht22.Read_DHT22_data()
		self.AS312=AS312(17)
		self.motion_dected =AS312.read_state()
		self.DF0026 =DF0026()
		self.light_value=self.DF0026.Read_data()
		self.fname="sensor_data.csv"
	def write_append_csv(self):
		data = { "Timestamp" : self.timestamp,
			"Temperature(oc)" : self.Temperature,
			"Humidity(%)" : self.humidity,
			"Light(lux)" :self.light_value,
			"Motion Dected": self.motion_dected
			}
		df = pd.DataFrame(data)
		if glob.glob(self.fname):	
			df.to_csv(self.fname,mode='a' ,index=False,header=False)
		else:
			df.to_csv(self.fname,mode='w' ,index=False)
class Memory_tester():
	def __init__(self):
		self.units={"K":10e3,"M": 10e6,"G":10e9}
		self.regex ="\d{4}\.\[0-9]{1,3}[K,M,G]"
		self.fname="../bash_scrpits/memorytest.sh" 
		self.output_bash=subprocess.check_output(["bash",self.fname],universal_newlines=True)
	def check_memory(self):
		try:
			if re.search(self.regex,self.output_bash):
				value,unit=match.group(0).split()
				try:
					return float(value)*self.units[unit]
				except KeyError:
					raise ValueError(f"unknown unit: {unit}")
			
		except subprocess.CalledProcessError as e:
			raise ValueError(f"Error running script:{e.output}")
	def error_check(self):
		mem=self.check_memory()
		max=32*10e9
		if mem >= 0.2* max:
			raise MemoryError("memory on pi is about to  used up")
		
\end{lstlisting}
