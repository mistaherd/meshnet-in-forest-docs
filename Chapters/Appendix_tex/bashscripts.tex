\section{Bashscripts}
in this section we  will have  the following bash files:
\begin{enumerate}
    \item Camerea
    \item main
    \item memorytest
    \item radiomodule
\end{enumerate}
\subsection{Camerea}
\begin{lstlisting}[style=bashstyle,caption={Code for triggering the camerea}]
    #!/bin/bash
    timestamp=$(date +"%Y-%m-%d_%H_%M_%S")
    fname="camera_output_$timestamp.png"
    output_dir="Images_camera"
    if [ ! -d "$output_dir" ]; then
    # Create the directory if it doesn't exist
    mkdir -p "$output_dir"
    fi
    rpicam-still --raw -o "$output_dir/$fname"
\end{lstlisting}
\newpage
\subsection{Main}
\begin{lstlisting}[style=bashstyle,caption={Code for runing the main function}]
    #!/bin/bash
    is_root() {
    if [[ $EUID -ne 0 ]]; then
        echo "This script requires root privileges. Please run with sudo."
        exit 1
    fi
    }
    if [[ $1 -eq 0 ]]; then
        echo "Error no arugments provided"
        echo -e "enter what is transmited:\n\r1:hello world \n\r2:text file \n\r3:csv file\n\r4:PNG\n\r"
        exit 1
    fi
    # Call the is_root function to verify permissions
    is_root
    sudo chmod g+rw /dev/ttyS0
    #get current time
    current_time=$(date +%H:%M)
    current_hour=$(echo $current_time | cut -d: -f 1)
    previous_hour=$((current_hour-1))
    while [ $current_time != "12:00" ]&&[ $current_time != "9:00" ]; do
    if [ $current_time == "$current_hour:00" ]; then
        # python Documents/Github/meshnetwork_in_forest/main/main.py $1
        python main/main.py $1
        echo "file ran successfully"
    fi
    break #because everyone needs a break sometime 
    done
\end{lstlisting}
\newpage
\subsection{Radio Module}
\begin{lstlisting}[style=bashstyle,caption={Code for the testing serial of the radio module}]
    #!/bin/bash
    #only use this  for  transcive module
    # Function to check if the script is run with root privileges
    is_root() {
    if [[ $EUID -ne 0 ]]; then
        echo "This script requires root privileges. Please run with sudo."
        exit 1
    fi
    }
    # Call the is_root function to verify permissions
    is_root
    # Set appropriate permissions for /dev/ttyS0 (consider group or user access)

    sudo chmod g+rw /dev/ttyS0

    if [[ "$1" == "1" ]]; then
    python test_tranmiter.py
    elif [[ "$1" == "0" ]]; then
    python test_reciver.py
    else
    echo "Invalid argument. Please provide 1 (transmitter) or 0 (receiver)."
    exit 1
    fi
\end{lstlisting}