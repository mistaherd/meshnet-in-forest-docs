
	now that we have established the  essential Hardware needed for our project we must consider the following  for the software of the  project:
	\begin{enumerate}
	    \item How to structure code 
	    \item Linux set up of sever and nodes
	    \item How will data be sent
	    \item Is this gonna be an OOP or functional  approach
	    \item How to program each device?
	\end{enumerate}
	\subsection{Raspberry pi OS}
	\label{pi os}
	In this Section  we want to keep in mind  that each os is  heavy wegiht  we need to consider the following:
	\begin{enumerate}
		\item When we  format the os is the  data on the sd lost does it croupt the  card?
		\item We want an os that is low in capacity ?
		\item Is desktop needed or can we  use  the  terminal?
		\item How does  the os  respond to usb drives
	\end{enumerate}
	According to the \cite{projects} the imager will erase all the data while installing the os, from reaserch  the suggestion of backing up the data is  a  good sugestion.
	
	\subsection{Sensor code}
	in this section, I will discuss the following :
	\begin{enumerate}
		\item DHT22
		\item AS312
		\item MCP3008
		\item DFR0026
		\item Kuman for Raspberry Pi 3B+ TFT LCD Display
		\item Raspberry Pi VR 220 Camera 
	\end{enumerate}
	this project code will mainly be object-oriented. so the goal is to first test it with my laptop and  Create A bash file  full of  commands to install  the Libraries
	making the code to be split up  into  different parts so that all that is needed is the Libraries I make and code that won't all have to  compiled in one file

	\subsubsection{DHT22}
	in this section we have to consider the following: 
	\begin{enumerate}
		\item the GPIO port as on page \pageref{Sychematic for DHT22 revised} this is connected to port 3 
		\item the type of output is digital so no  extra hardware/code is needed
	\end{enumerate}
	the following is  a rough guide on how to read from the DHT22 I got this from the following \href{https://www.instructables.com/Raspberry-Pi-Tutorial-How-to-Use-the-DHT-22/}{link}
	(\textbf{Note: I haven't tested this due to the constraints of this year so I can only go off what others have done.} )Firstly open the terminal in the pi and
	type the following commands
	\begin{verbatim}
		git clone https://github.com/adafruit/Adafruit_Python_DHT.git
		cd Adafruit_Python_DHT
		sudo apt-get update
		sudo apt-get install build-essential python-dev
		sudo python setup.py install

	\end{verbatim}
	\newpage
	the following is the sample code  for this sensor 
	\begin{lstlisting}[style=mystyle,caption={Example code for DHT2}]
		#Libraries
		import Adafruit_DHT as dht
		from time import sleep
		def setup_DHT22(Gpoiport:int):
		humidityy,temp=dht.read_retry(dht.DHT22, Gpoiport)
			sleep(5)
			return humundity,temp
		h,t=setup_DHT22(3)
		print('Temp={0:0.1f}*C  Humidity={1:0.1f}%'.format(t,h))
	\end{lstlisting}
	this code will  import the DHT Libraries from Adafruit and  the  module of time, next a function is called where the  port is defined and  in the DHT  is mainly used DHT Libraries 
	
	\subsubsection{AS312}
	for this section i followed  this \href{https://pimylifeup.com/raspberry-pi-motion-sensor/}{link}
	we also want to keep in mind the following:
	\begin{enumerate}
		\item This has a  digital interface and is connected  to GPIO 27 
	\end{enumerate}
	Here are the rough steps firstly  type the following into the  terminal 
	\begin{verbatim}
		sudo apt-get install python-rpi.gpio
	\end{verbatim}
	\newpage
	Then type this into an IDE of your  choosing
	\begin{lstlisting}[style=mystyle,caption={Example code for AS312}]
		import RPi.GPIO as GPIO
		import time

		pir_sensor = 27
		GPIO.setmode(GPIO.BOARD)

		GPIO.setup(pir_sensor, GPIO.IN)
		current_state = 0
		
		time.sleep(0.1)
		current_state = GPIO.input(pir_sensor)
		if current_state == 1:
			print("GPIO pin %s is %s" % (pir_sensor, current_state))
			# trigger camera
		# must look up this 
		GPIO.cleanup()
	\end{lstlisting}
	
	\subsubsection{DFR0026}	
	from the last example, nothing has changed from the last component
	an example code for this can be found on page \pageref{adc code}
	\subsection{MCP3008}
	for this section, we want to consider the following:
	\begin{enumerate}
		\item The MCP3008 data out is GIPO 9 
	\end{enumerate}
	this section follows this \href{https://randomnerdtutorials.com/raspberry-pi-analog-inputs-python-mcp3008/}{link}
	firstly try the following in command in the terminal 
	\begin{verbatim}
	sudo raspi-config nonint do_spi 0
	\end{verbatim}
	
	\label{adc code}
	\begin{lstlisting}[style=mystyle,caption={ADC code}]
		from gpiozero import MCP3008
		from time import sleep
		DFR0026 = MCP3008(channel=0, device=0,port=9)
	
		print ('raw: {:.5f}'.format(DFR0026.value))
		sleep(0.1)
	\end{lstlisting}
	\subsection{Kuman for Raspberry Pi 3B+ TFT LCD Display}
	first in the terminal type the following:
	\begin{verbatim}
		sudo pip install Adafruit_ILI9341
		sudo pip install Pillow              
	\end{verbatim}
	
	\begin{lstlisting}[style=mystyle,caption={example code for LCD display }]
		import Adafruit_ILI9341 as ILI9341
		from PIL import Image, ImageDraw, ImageFont
		# Configure the display driver GPIO pins
		disp = ILI9341.ILI9341(dc=25, spi=SPI.SpiDev(0, 0), rst=24, mosi=19, sclk=18, miso=21)
		# Initialize the display
		disp.begin()
		# Set the font
		font = ImageFont.truetype('Arial.ttf', 16)
		# Set the text colour
		text_color = (255, 255, 255)  # White text color
		# Create a blank image
		image = Image. new('RGB', (disp.width, disp.height), (0, 0, 0))
		# Create an ImageDraw object for drawing on the image
		draw = ImageDraw.Draw(image)

		# Draw the text onto the image
		draw.text((10, 10), 'Hello, World!', font=font, fill=text_color)
		disp.display(image.convert('RGB'))
	\end{lstlisting}
	\subsection{Raspberry Pi VR 220 Camera}
	to get started with this simply look at the following \href{https://projects.raspberrypi.org/en/projects/getting-started-with-picamera/4}{link}
	here is an example of the code of this module :
	\begin{lstlisting}[style=mystyle,caption={example code for camera}]
		from picamera import PiCamera
		from time import sleep

		camera = PiCamera()

		camera.start_preview()
		sleep(5)
		camera.stop_preview()
	\end{lstlisting}

	\subsection{MM2 Series 900 MHz}
	for this section, the seller of this module has no public  documentation so it is hard to  come  with  an make a  interface for  this section   
	\subsection{code structure}
	The code structure for this  will be an object-oriented program all the individual sensors and  hardware  for the pi will be as displayed above the code in this section will be formatted into objects for example I will have an  object   called proj\_sensor and  a method of this  would be  DHT22 while an attribute of this would  be  the  sample rate
	the following is a  rough breakdown of the  structure of the code
	\begin{itemize}
		\item Sensor object
		
		\begin{itemize}
			\item Temperature and humanity method
			\item light method
			\item Motion method which triggers the camera
			\item Battery method which is a constructor method
			\item Memory method which  links with the radio 
		
		\end{itemize}
		
		\item radio object which reads from Memory and  transmits the data 
	
	\end{itemize}
	\subsection{File structure}
	For the  File structure, we want our sensor data to be stored every hour in a  CSV file with the following column headings:
	\begin{enumerate}
		\item timestamp
		\item Heat
		\item Humidity
		\item light level
		\item motion detected (True/False)
	\end{enumerate}
	for the writing to Date, we will use Pandas to write to the CSV file
	for file sorting, I will use the Python Library glob  which I can use  to look for  files 
	the following is an example of how  to  make a CSV file:
	firstly let's make a data frame:
	\begin{lstlisting}[style=mystyle,caption={sample code for turning sensor data into a data}]
		import pandas as pd
		import numpy as np
		from datetime import datetime
		cols_name=["Timestamp","Tempeature","Hummidty","Light_level","Motion_dected"]

		#assume that being recorded now
		data=[]
		timestap=datetime.now()
		timestap=timestap.strftime("%d/%m/%Y %H:%M:%S")
		Current_state=1
		Heat=0.40
		Hummidty=1.0
		Light_level=0.23
		data=np.array([[timestap],[Heat],[Hummidty],[Light_level],[Current_state]])
		data=data.T
		df= pd.DataFrame(data,columns=cols_name)
	\end{lstlisting}
	Next, use the.To\_csv method from  pandas
	another Libraries that could  be useful is the Tkinter
	here is a  sample of how to   store where the  file is  gonna be:
	\begin{lstlisting}[style=mystyle,caption={example code for storing directory}]
		import tkinter as tk
		from tkinter import filedialog
		import json
		import os

		root = tk.Tk()
		root.withdraw()
		selected_dir = filedialog.askdirectory()

		if not os.path.exists('selected_dir.json'):
			# Write the selected directory to a JSON file
			with open('selected_dir.json', 'w') as f:
				json.dump(selected_dir, f)
				print("Successfully saved selected directory to JSON file.")
		else:
			print("File 'selected_dir.json' already exists. Not saving the directory.")

		root.quit()
	\end{lstlisting}
	Other useful Libraries allow you to  select all  .csv, png  called glob
	\subsection{Test Driven devolmponet}
	In this  project ill will be useing  Test Driven Development (TDD) is a software development approach where tests are written before the actual code
	the following is the advantages and dissadvantges of TDD
	\begin{enumerate}
		\item 
	\end{enumerate}
	% \subsection{Conclusion}