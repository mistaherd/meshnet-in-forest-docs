
Having established the essential hardware needed for this project. Next, consider the following  for the software of the  project:
	\begin{enumerate}
	    \item How to structure code 
	    \item Linux set up of sever and nodes
	    \item How will data be sent
	    \item Will this be an OOP or functional approach?
	    \item How to program each device?
	\end{enumerate}
	\subsection{Raspberry Pi OS}
	\label{pi os}
	In this section, it must be kept in mind  that each OS is  heavyweight the following needs to be considered:
	\begin{enumerate}
		\item If the SD is formatted the  data on the SD is lost. Does it corrupt the card?
		\item An os that is low in capacity 
		\item Is a desktop needed or can we use the  terminal?
		\item How does  the OS  respond to USB drives?
	\end{enumerate}

	According to the \cite{projects} the imager will erase all the data while installing the os. From research  the suggestion of backing up the data is  a  good suggestions.
	for now the recommended OS is used. and  strip down as the project progresses. which will be  discussed  in the methodology section of this report.
	
	\subsection{Sensor code}
	In this section the following will be discussed :
	\begin{enumerate}
		\item DHT22
		\item AS312
		\item MCP3008
		\item DFR0026
		\item Kuman for Raspberry Pi 3B+ TFT LCD Display
		\item Raspberry Pi VR 220 Camera 
	\end{enumerate}

	the project code will mainly be object-oriented. so the goal is to first test it with my laptop and  Create A bash file  full of  commands to install  the libraries,making the code to be split up  into  different parts so that all that is needed is the libraries used and code that won't all have to  compiled in one file.

	\subsubsection{DHT22}
	In this section we have to consider the following: 
	\begin{enumerate}
		\item The GPIO port as on page \pageref{Sychematic for DHT22 revised} This is connected to port 3 
		\item The type of output is digital so no  extra hardware/code is needed
	\end{enumerate}

	The following is  a rough guide on how to read from the DHT22 from the following \href{https://www.instructables.com/Raspberry-Pi-Tutorial-How-to-Use-the-DHT-22/}{link}.
	Firstly open the terminal in the Pi and
	type the following commands:
	\begin{lstlisting}[style=bashstyle]
		git clone https://github.com/adafruit/Adafruit_Python_DHT.git
		cd Adafruit_Python_DHT
		sudo apt-get update
		sudo apt-get install build-essential python-dev
		sudo python setup.py install
	\end{lstlisting}
	the code does the following:
	\begin{enumerate}
		\item firstly git clone will clone the  repository on to device
		\item Then change dirertorys  a
		\item update linux
		\item install dev kit for  python 
		\item and install the setup 
	\end{enumerate}
	
	\newpage
	this will then lead to  the  following code:
	\begin{lstlisting}[style=mystyle,caption={Example code for DHT2},numbers=left,firstnumber=1]
		#Libraries
		import Adafruit_DHT as dht
		from time import sleep
		def setup_DHT22(Gpoiport:int):
		humidityy,temp=dht.read_retry(dht.DHT22, Gpoiport)
			sleep(5)
			return humundity,temp
		h,t=setup_DHT22(3)
		print('Temp={0:0.1f}*C  Humidity={1:0.1f}%'.format(t,h))
	\end{lstlisting}
	this code will do the following:
	\begin{enumerate}
		\item Import DHT from the adafuit library
		\item in the  function which takes the gpioport  as an integer this will read the data on the pin and  print it  out
	\end{enumerate}
	
	\subsubsection{AS312}
	for this section i followed  this \href{https://pimylifeup.com/raspberry-pi-motion-sensor/}{link}
	we also want to keep in mind the following:
	\begin{enumerate}
		\item This has a  digital interface and is connected  to GPIO 27 
	\end{enumerate}
	Here are the rough steps firstly  type the following into the  terminal 
	\begin{verbatim}
		sudo apt-get install python-rpi.gpio
	\end{verbatim}
	which will  intall a gpio python module
	\newpage
	Then type this into an IDE of your  choosing
	\begin{lstlisting}[style=mystyle,caption={Example code for AS312},numbers=left,firstnumber=1]
		import RPi.GPIO as GPIO
		import time

		pir_sensor = 27
		GPIO.setmode(GPIO.BOARD)

		GPIO.setup(pir_sensor, GPIO.IN)
		current_state = 0
		
		time.sleep(0.1)
		current_state = GPIO.input(pir_sensor)
		if current_state == 1:
			print("GPIO pin %s is %s" % (pir_sensor, current_state))
			# trigger camera
		# must look up this 
		GPIO.cleanup()
	\end{lstlisting}
	this code does the  following:
	\begin{enumerate}
		\item it will look at the  pin for a pulse 
		\item onece it sences a pulse  it will tiggerr  the camerae
	\end{enumerate}
	\subsubsection{DFR0026}	
	from the last example, nothing has changed from the last component
	an example code for this can be found on page \pageref{adc code}
	\subsection{MCP3008}
	for this section, we want to consider the following:
	\begin{enumerate}
		\item The MCP3008 data out is GIPO 9 
	\end{enumerate}
	this section follows this \href{https://randomnerdtutorials.com/raspberry-pi-analog-inputs-python-mcp3008/}{link}
	firstly try the following in command in the terminal 
	\begin{verbatim}
	sudo raspi-config nonint do_spi 0
	\end{verbatim}
	
	\label{adc code}
	\begin{lstlisting}[style=mystyle,caption={ADC code},numbers=left,firstnumber=1]
		from gpiozero import MCP3008
		from time import sleep
		DFR0026 = MCP3008(channel=0, device=0,port=9)
	
		print ('raw: {:.5f}'.format(DFR0026.value))
		sleep(0.1)
	\end{lstlisting}
	this code will  selcect a  channel and device , port and  print the vaules of the adc's
	\subsection{Raspberry Pi VR 220 Camera}
	to get started with this simply look at the following \href{https://projects.raspberrypi.org/en/projects/getting-started-with-picamera/4}{link}
	here is an example of the code of this module :
	\begin{lstlisting}[style=mystyle,caption={example code for camera},numbers=left,firstnumber=1]
		from picamera import PiCamera
		from time import sleep

		camera = PiCamera()

		camera.start_preview()
		sleep(5)
		camera.stop_preview()
	\end{lstlisting}
	this will  take a photo of what is in fron of the  camera

	\subsection{MM2 Series 900 MHz}
	for this section, the seller of this module has no public  documentation so it is hard to  come  with  an make a  interface for  this section   
	\subsection{code structure}
	The code structure for this  will be an object-oriented program all the individual sensors and  hardware  for the pi will be as displayed above the code in this section will be formatted into objects for example I will have an  object   called proj\_sensor and  a method of this  would be  DHT22 while an attribute of this would  be  the  sample rate
	the following is a  rough breakdown of the  structure of the code
	\begin{itemize}
		\item Sensor object
		
		\begin{itemize}
			\item Temperature and humanity method
			\item light method
			\item Motion method which triggers the camera
			\item Battery method which is a constructor method
			\item Memory method which  links with the radio 
		
		\end{itemize}
		
		\item radio object which reads from Memory and  transmits the data 
	
	\end{itemize}
	\subsection{File structure}
	For the  File structure, we want our sensor data to be stored every hour in a  CSV file with the following column headings:
	\begin{enumerate}
		\item timestamp
		\item Heat
		\item Humidity
		\item light level
		\item motion detected (True/False)
	\end{enumerate}
	for the writing to Date, we will use Pandas to write to the CSV file
	for file sorting, I will use the Python Library glob  which I can use  to look for  files 
	the following is an example of how  to  make a CSV file:
	firstly let's make a data frame:
	\begin{lstlisting}[style=mystyle,caption={sample code for turning sensor data into a data},numbers=left,firstnumber=1]
		import pandas as pd
		import numpy as np
		from datetime import datetime
		cols_name=["Timestamp","Tempeature","Hummidty","Light_level","Motion_dected"]

		#assume that being recorded now
		data=[]
		timestap=datetime.now()
		timestap=timestap.strftime("%d/%m/%Y %H:%M:%S")
		Current_state=1
		Heat=0.40
		Hummidty=1.0
		Light_level=0.23
		data=np.array([[timestap],[Heat],[Hummidty],[Light_level],[Current_state]])
		data=data.T
		df= pd.DataFrame(data,columns=cols_name)
	\end{lstlisting}
	Next, use the.To\_csv method from  pandas
	another Libraries that could  be useful is the Tkinter
	here is a  sample of how to   store where the  file is  gonna be:
	\begin{lstlisting}[style=mystyle,caption={example code for storing directory},numbers=left,firstnumber=1]
		import tkinter as tk
		from tkinter import filedialog
		import json
		import os

		root = tk.Tk()
		root.withdraw()
		selected_dir = filedialog.askdirectory()

		if not os.path.exists('selected_dir.json'):
			# Write the selected directory to a JSON file
			with open('selected_dir.json', 'w') as f:
				json.dump(selected_dir, f)
				print("Successfully saved selected directory to JSON file.")
		else:
			print("File 'selected_dir.json' already exists. Not saving the directory.")

		root.quit()
	\end{lstlisting}
	Other useful Libraries allow you to  select all  .csv, png  called glob
	for our TDD Section we will have  use the  following command:
	\label{TDD sample bash}
	\begin{verbatim}
		# !/bin/bash

		dir_name=$1

		size=$(du -sh "$dir_name" | cut -f1)

		echo "Directory size: $size"
	\end{verbatim}
	This is a script that will look at a  director this can be home directory  this will cal the  space if 13K
	the "| cut -f1" will only foucs on the  size string messeage and then print out the  size. this is  just  a sample  script 
	\subsection{Test Driven development}
	In this  project ill will be useing  Test Driven Development (TDD) is a software development approach where tests are written before the actual code
	the following is the advantages of TDD:
	\begin{enumerate}
		\item Advantages
		\begin{enumerate}
			\item TDD forces you to consider potential failure points and edge cases upfront, leading to earlier detection and resolution of bugs.
			\item TDD encourages you to think about the desired behaviour and interfaces of your code
			\item TDD provides immediate feedback on whether your code works as intended,
		\end{enumerate}
	\end{enumerate}

	% \subsection{Conclusion}