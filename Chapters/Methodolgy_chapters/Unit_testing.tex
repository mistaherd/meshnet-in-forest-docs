The following is the process of devlop:
\begin{itemize}
    \item To make  test  that will be  there for  the  codeing  section of  the  project 
\end{itemize}
this section will discuss the  following for  testing:
\begin{enumerate}
    \item 1 x DHT22
    \item 1 x DFR0026
    \item 1 x AS312
    \item 1 x SB Components LoRa HAT for Raspberry Pi
    \item 1 x MCP3008
    \item 1 x Raspberry Pi VR 220 Camera
    \item  1 x Li-polymer Battery HAT 
\end{enumerate}
\subsubsection{DHT22}
According to the  data sheet \cite{sparkfun} seen as  the data is   8 bits  and  the  range at which this   operates at  -40 to 80$^{o}$c for tempeature
meaning we have  at least  7 bit in the  exponent to  represent  the   measured value.
to represent  the  high  end of this  sensor i used the  following calculation:
$$ 2^6 +2^4 = 80$$ which mean  we have  a 2 bits dedicated to decimal place so the  high temperature to be 80.3$^{o}$c
for the  lowest temp we have 6 bits  to  represent  - 40 due to  2s complement  so lowest  will   be -40.3$^{o}C$
so with that  that  stablish we  must  make a  unit  that will do  the  following:
\begin{enumerate}
    \item Test if the  output is  a float
    \item Test the  high end of  the  temp sensor so it  reads  80.3 as  the highest
    \item Test for the  lowest   temp   around 
\end{enumerate}
be sure to follow  steps  for  folder  setup  follow instructions on page \pageref{folderstructure}.
we get the following sample code:
\begin{lstlisting}[style=mystyle,caption={sample test intial code}]
import unittest
from protest import Read_DHT22
class test_project_code(unittest.TestCase):
    def test_DHT_22_temp_output_type(self):
        self.assertIsInstance(Read_DHT22, float)
    def test_DHT22_temp_range(self):
        self.assertGreaterEqual(Read_DHT22,-30.3)
        self.assertLessEqual(Read_DHT22,80.3)
    
\end{lstlisting} 
This code imports unittest . the from DHT22 is  a python files  we can  install  functions from other python files this can be useful for testing purposes
then we initialized a test class call Unittest.test as our first function of the class  we  check if the number of the output  is a float or not this is  for  testing  temperature
the next function we test for  is the range look at the data sheet online  for . This code is  simply testing the  limits of the  DHT22
for  humidity  the Data sheet which ranges from 0 to 100 \%
we want to test for the following:
\begin{enumerate}
    \item Test if the recorded output is a tuple
    \item Test if the temperature recorded  is a float or  integer
    \item Test if the temperature recorded is  in the  range from -30 to 80.3
    \item Test if the humidity recorded is  a  float or  integer
    \item Test if the humidity recorded is  between 0 and 100
\end{enumerate}
this lead to the  following code 
\begin{lstlisting}[style=mystyle,caption={sample test for DHT22}]
import unittest
from DHT22 import DHT22
dht22_instance=DHT22()
class test_project_code(unittest.TestCase):
    hum,temp=Read_DHT22(2)
    def test_DHT22_output_type(self):
        self.assertIsInstance(dht22_instance.Read_DHT22_data, tuple)

    def test_DHT_22_temp_output_type(self):
        self.assertIsInstance(temp, (int,float) )

    def test_DHT22_temp_range(self):
        self.assertGreaterEqual(temp,-30.3)
        self.assertLessEqual(temp,80.3)
\end{lstlisting}
seen as we expect our sensor to  print out a humidity and temp values we  set the  output to  a tuple 
to test for this we use isInstacne which will test if its a tuple
next we test for the  limits of the  humidity
\subsubsection{DFR0026}
According to the  data sheet \cite{ada} we must keep in mind  that this  component is  connected to  an ADC 
this  will  give  me  the  following  test conditions:
\begin{enumerate}
    \item Test if  the output is a dictionary with elements of a string and intger in it.

    \item Test  the  range of this  with the  upper limit being 5v the analogue voltage meaning:$\begin{aligned}\text{output}&= \frac{2^n\cdot \text{Analogue Input value}}{\text{Refeence voltage}}\\ &=\frac{2^{16}\cdot 5}{5}=65536\end{aligned}$
    \item test the  lover limit being 3.3v as the analogue$\frac{2^{16}\codt 3.3}{5}=43253$
\end{enumerate}
\begin{lstlisting}[style=mystyle,caption={unit test for  DFR0026 and  MCP3008}]
    import unittest
    from DFR0026 import DFR0026
    class test_project_code(unittest.TestCase):
    def test_DFR0026_out_type(self):
        self.assertIsInstance(DFR0026().read_voltage(),dict[str,int])
    def test_DFR0026_out_range(self):
        self.assertLessEqual(DFR0026().read_voltage(),65536)
        self.assertGreaterEqual(DFR0026().read_voltage(),43253)
\end{lstlisting}
\subsubsection{AS312}
for  this section  we  want our  tests  to  be  the following:
\begin{enumerate}
    \item test if output type is boolean 
\end{enumerate}
we can  now add to the snipppet :
\begin{lstlisting}[style=mystyle,caption={unit test for AS312}]
    import unittest
    from AS312 import AS312
    AS312_instance=AS312()
    class test_project_code(unittest.TestCase):
    def test_AS312_out_type(self):
        self.assertIsInstance(AS312_instance.read_state,booll)
\end{lstlisting}
seen as  this is a motion sensor  our output will be true or false.

\subsubsection{Raspberry Pi VR 220 Camera}
according to the  data sheet \cite{Camera} The  resolution to  it uses is  1080p50 which is 1920x1080p so our  tests will have to  in corporate  the  following:
\begin{enumerate}
    \item Test  if the file can run
\end{enumerate}
This would lead to the following code snippet.
\begin{lstlisting}[style=mystyle,caption={camera unit test}]
    def test_Raspberry_Pi_VR220_out_shape(self):
        self.assertEqual(camera_obj.run.returncode, 0)
\end{lstlisting}
This function check the pixel count or resolution

\subsubsection{memory module}
in this section will discuss the following:
\begin{enumerate}
    \item silicon power 32GB 
For this use a  bash script(see this on page \pageref{TDD sample bash}) to test the size in a  certain range for the silicon  SD card
    \item which will be 0B to 32GB
    
    \begin{lstlisting}[style=mystyle]
        def Test_memory_silicon_power_32GB(self):
            self.assertLessEqual(memorytest_obj.check_memory,32e9)
            self.assertGreaterEqual(memorytest_obj.check_memory,0)
    \end{lstlisting}
\end{enumerate} 
\subsubsection{SB Components LoRa HAT}
for this section we want to test the following:
\begin{enumerate}
    \item Test the serial connection
    \item Test the serial interrupt
    \item Test the sending/receiving of a message like "hello world"
    \item Test the sending/receiving of a text file 
    \item Test the sending/receiving of a csv file
    \item Test the sending/receiving of a image file
\end{enumerate}
This will give  the following code:
\begin{lstlisting}[style=mystyle]
    import unittest
    from Radiomodule import Transciever
    Transciever_instance=Transciever()
    class test_project_code(unittest.TestCase):
        def test_serial_connection(self):
            self.assertIsInstance(Transciever_instance.transceive_ser,serial.Serial)
        def test_serial_interrupt(self):
            self.assertEqual(Transciever_instance.event.is_set(),(False,True))
        def test_transceiver_test_message(self):
            message=Transciever_instance.message
            Transciever_instance.transceive_test_message(True)
            received_message=Transciever_instance.transceive_test_message(False)
            self.assertEqual(message,received_message)
        def test_transceiver_test_txt_file(self):
            txt_fname=Transciever_instance.txt_fname
            with open(txt_file,'r') as f:
                expected_txt=f.read()
            Transciever_instance.transceive_test_txt_file(True)
            received_txt_file=Transciever_instance.transceive_test_txt_file(False)
            self.assertEqual(expected_txt,received_txt_file)
        def test_transciver_test_csv(self):
            csv_fname=Transciever_instance.csv_fname
            expected_df=pd.read_csv(csv_fname)
            Transciever_instance.transceive_test_csv(True)
            reviced_df=Transciever_instance.transceive_test_csv(False)
            self.assertEqual(expected_df,reviced_df)
        def test_trancsive_img_file(self):
            img_fname=Transciever_instance.png_fname
            with open(img_fname,'rb')as f:
                expted_out=f.read()
            Transciever_instance.Transcevie_png_file(True)
            received_bin=Transciever_instance.Transcevie_png_file(False)
            self.assertEqual(expted_out,received_bin)
    if __name__ == '__main__':
        unittest.main()
\end{lstlisting}
\subsubsection{Unit test iterations}
This section will discuss the iterations of unit test the following are the iterations of  our unit testing:
\begin{enumerate}
    \item In the first iteration  the following was worked on:
    \begin{itemize}
        \item DHT22
        \item DFR0026
        \item AS312
        \item Camera
        \item turbo 1GB
        \item silicon power 32Gb
    \end{itemize}

    This was formed in the when the lit review was written
    \item most the iterations where coming out the code and testing  each section 
\end{enumerate}
