The following is the process of devlop:
\begin{itemize}
    \item To make  test  that will be  there for  the  codeing  section of  the  project 
\end{itemize}
this section will discuss the  following for  testing:
\begin{enumerate}
    \item 1 x DHT22
    \item 1 x DFR0026
    \item 1 x AS312
    \item 1 x SB Components LoRa HAT for Raspberry Pi
    \item 1 x MCP3008
    \item 1 x Raspberry Pi VR 220 Camera
    \item  1 x Li-polymer Battery HAT 
\end{enumerate}
\subsubsection{DHT22}
According to the  data sheet \cite{sparkfun} seen as  the data is   8 bits  and  the  range at which this   operates at  -40 to 80$^{o}$c for tempeature
meaning we have  at least  7 bit in the  exponent to  represent  the   measured value.
to represent  the  high  end of this  sensor i used the  following calculation:
$$ 2^6 +2^4 = 80$$ which mean  we have  a 2 bits dedicated to decimal place so the  high temperature to be 80.3$^{o}$c
for the  lowest temp we have 6 bits  to  represent  - 40 due to  2s complement  so lowest  will   be -40.3$^{o}C$
so with that  that  stablish we  must  make a  unit  that will do  the  following:
\begin{enumerate}
    \item Test if the  output is  a float
    \item Test the  high end of  the  temp sensor so it  reads  80.3 as  the highest
    \item Test for the  lowest   temp   around 
\end{enumerate}
be sure to follow  steps  for  folder  setup  follow instructions on page \pageref{folderstructure}.
we get the following sample code:
\begin{lstlisting}[style=mystyle,caption={sample test intial code}]
import unittest
from protest import Read_DHT22
class test_project_code(unittest.TestCase):
    def test_DHT_22_temp_output_type(self):
        self.assertIsInstance(Read_DHT22, float)
    def test_DHT22_temp_range(self):
        self.assertGreaterEqual(Read_DHT22,-30.3)
        self.assertLessEqual(Read_DHT22,80.3)
    
\end{lstlisting} 
This code import unitest . the from protest is  a python files  we can  install  functions from other python files this can be usefull for testing purposes
then we initalized a test class call Unittest.testcase our firstion fucntion of the class  we  check if the number of the output  is a float or not this is  for  testing  tempearture
the next function we test for  is the range i look at the datasheet online   this code is  simpley testing the  limits of the  DHT22
for  humidity  the Datasheet which ranges from 0 to 100 \%
we want to test for the following:
\begin{enumerate}
    \item Test if the  output is  a float
    \item Test if the  output ranges 0 to 100
\end{enumerate}
this lead to the  following code 
\begin{lstlisting}[style=mystyle,caption={sample test for DHT22}]
import unittest
from protest import Read_DHT22
class test_project_code(unittest.TestCase):
    hum,temp=Read_DHT22(2)
    def test_DHT22_output_type(self):
        self.assertIsInstance(Read_DHT22,tuple)
    #....

    def test_DHT22_hum_output_type(self):
        self.assertIsInstance(hum,float)

    def test_DHT22_hum_range(self):
        self.assertGreaterEqual(hum,0.0)
        self.assertLessEqual(hum,100.0)
\end{lstlisting}
seen as we expect our sensor to  print out a humdity and temp values we  set the  output to  a tuple 
to test for this we use isInstacne which will test if its a tuple
next we test for the  limits of the  humidity
\subsubsection{DFR0026 \& MCP3008}
According to the  datasheet \cite{ada} we must keep in mind  that this  componet is  connected to  an ADC 
this  will  give  me  the  following  test conditions:
\begin{enumerate}
    \item Test if  the output is a float

    \item Test  the  range of this  with the  upper limit being 5v 
    \item test the  lover limit being 0 
\end{enumerate}
\begin{lstlisting}[style=mystyle,caption={unit test for  DFR0026 and  MCP3008}]
    import unittest
    from protest import Read_DHT22,Read_MCP3008
    class test_project_code(unittest.TestCase):
    def test_DFR0026_MCP3008_out_type(self):
        self.assertIsInstance(Read_MCP3008,float)
    def test_DFR0026_MCP3008_out_range(self):
        self.assertLessEqual(5.0000000)
        self.assertGreaterEqual(0.0000000)
\end{lstlisting}
this code is in the same in theres of limits
\subsubsection{AS312}
for  this section  we  want our  tests  to  be  the following:
\begin{enumerate}
    \item test for type is boolean 
\end{enumerate}
we can  now add to the snipppet :
\begin{lstlisting}[style=mystyle,caption={unit test for AS312}]
    def test_AS312_out_type(self):
        self.assertIsInstance(Read_AS312,bool)
\end{lstlisting}
\textbf{Note : Don't forget to import read_as12 function from test file}
seen as  thhis is a motion sensor  our ouout will be true or false
\subsubsection{Raspberry Pi VR 220 Camera}
according to the  data sheet \cite{Camera} The  resolution to  it uses is  1080p50 which is 1920x1080p so our  tests will have to  in corporate  the  following:
\begin{enumerate}
    \item Test the  output shape  if open cv is  gonna  be  used 
    \begin{enumerate}
        \item Test  the  amount of   elements in the  3 dimensional   array 
    \end{enumerate}
    \item Test the  file  type  is png
\end{enumerate}
This would lead to the following code snippet.
\begin{lstlisting}[style=mystyle,caption={camera unit test}]
    def test_Raspberry_Pi_VR220_out_shape(self):
        self.assertEqual(Read_Raspberry_PiVR220.shape,(1920,1080,3))
\end{lstlisting}
This function check the pixel count or resolution

\subsubsection{memory module}
in this section will discuss the following:
\begin{enumerate}
    \item silicon power 32GB 
\end{enumerate}
For this use a  bash script(see this on page \pageref{TDD sample bash}) to test the size in a  certain range for the silicon  SD card
    \item silicon power 32GB
\end{enumerate} 
\subsubsection{SB Components LoRa HAT}
 
\subsubsection{Unit test iterations}
the first iterations as see here has the following problems for the sensors:
\begin{enumerate}
    \item Time stamp for DHT22 wasn't in a string format 
    \item Forget to look for but a float and int  in the DHT22.rgiead function
\end{enumerate}
