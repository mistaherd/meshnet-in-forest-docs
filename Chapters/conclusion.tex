\section{Conclusion}
\label{ch:conclusion} % For easy referencing

In this final year project, We have explored the implementation and development of mesh networks within the challenging environment of a forest. Through:
\begin{enumerate}
    \item The wireless commutation environment 
    \item The desired characteristics of the sensors
    \item The available electronics devices for this limitation
    \item The limited hardware avabilve
    \item The  fundamental software needed
    \item The Setup of the software
    \item The examination of the programming software needed
    \item The test before writing of any code 
    \item The discuss of how to record data
    \item The Limitations
\end{enumerate}
\subsection{Key Contributions}

This project has contributed significantly to the understanding to the process of developing mesh networks in forest settings by:

\begin{itemize}
    \item \textbf{Developing a Model:} In this we used serial communication to test the nature of  which the network will send/ receive data from the network 
    \item \textbf{Addressing Challenges:} in a line of sight environment the sending and receiving of image is a hard task due to  how large the file is byte by byte
    \item \textbf{Performance Evaluation:} every method eventually worked via serial but image files had the  most problems
    \item \textbf{Practical Implications:} Schedule the sending  and  receiving  of data we can use this to extract the sensor data
    \item \textbf{Familarilty  diffent tools:} This project forces the project implementation to use the terminal in linux ,bash , In Linux the concept of Dev files these are files where port can be defined for example port 80 on window would be 
\end{itemize}
\section{Sources of Error}
In this section the following will be discussed:
\begin{enumerate}
    \item Radio module
    \item Time management
    \item Lack of knowledge of linux OS
\end{enumerate}
\section{Future Work}

While this project hasn't achieved its core objectives, several avenues for future research and development remain open:

\begin{itemize}
    \item \textbf{Sockets:} The project could of ventrured into socket programming ,A socket is an endpoint of a two-way communication link between two programs running on the network.we picked serial communication for testing  but this is where the main server and client can defined.
    \item \textbf{Helpful Tools:}  Linux has a wide range of networking tools such as:
    \begin{itemize}
        \item\textbf{Nmap} is a network scanner designed to discover hosts and services on a computer network. It sends packets and analyzes the responses to gather information 
        \item \textbf{ifconfig} Which provides extensive control over interfaces, addresses and routes
        \item \textbf{traceroute} Traces the route packets take to reach a destination
        \item \textbf{route} Displays or manipulates the kernel's IP routing table
        \item \textbf{arp} Displays and manages the Address Resolution Protocol (ARP) cache, which maps IP addresses to MAC addresses
        \item \textbf{tcpdump} Captures and analyzes network traffic, useful for diagnosing network issues.
        \item \textbf{iftop} Displays a real-time bandwidth monitor for network interfaces.
    \end{itemize} 
    \item \textbf{Energy Efficiency:} Investigate further what can be done to make  our system more energy effective i.e piplineing the sensor data
    \item \textbf{Security:} Investigate how to make our data secure via rsa and different protocols
    \item \textbf{PCB board :} After testing the board one problem was wiring up the sensor a potential for making a custom  Pi hat that will provide an easy was of connecting  our sensor 
 
   
\end{itemize}

\subsection{Final Remarks}
The deployment of mesh networks in forest environments holds immense promise. The work presented here serves as a solid foundation for future endeavors.into the actual routeing of the network and testing this