\section{Introduction}
% C:\Users\User\OneDrive\Documents\latexfyp\Chapters\liteature review chapters\Introduction.tex
The following literature review explores mesh networks in a wooded area, When communicating from two devices across a network there are many issues associated with this communication such as signal loss due to:
	\begin{itemize}
		\item Environmental conditions such as rain .lighting etc
		\item Whether the device's antenna is in line of sight with each other
		\item If the devices are in the line of sight with each other. We can still reflect from a multi-path environment
		\item Possibility of falling trees obstructing the path of the signal causing more attenuation in the signal strength
	\end{itemize}
	This project aims to explore mesh networks and transmit data across them,  a mesh network is a type of network where no node in the network acts as a master. A node is a device that has a transceiver. As we look at the environment in which this project will be carried out, we can expect different phenomena to occur such as  Attenuation According to ITU \cite{ITU} "Attenuation due to vegetation varies widely due to the irregular Nature of the medium and the wide range of species, densities and water content obtained in practice"
	Transmitting any radio wave takes energy, Another factor to consider is whether wind will cause a delay in the signal. This report aims to show my findings and try to account for environmental conditions
	\subsection{Overview} \label{sec: overview}

		The following section provides a brief overview of  this project on mesh networks in a forest the following question is:
		
		\begin{enumerate}
			\item What frequencies can transmit in a forest
			\begin{itemize}
				\item What are the disadvantages of transmitting at this range
				\item What are the effects of the multi-path environment when there is a line of sight
				\item What happens to bon-line of sight
			\end{itemize}
			\item What sensors /senor modules should be used
			\begin{itemize}
				\item What sensors will give a good range in an Irish forest
				\item What are the limitations on the board used
				\item Is there any need for any additional hardware to  accommodate a specific board 
			\end{itemize}
			\item What microprocessor/hardware should be used?
			\begin{itemize}
				\item The advantages/disadvantages of  Arduino vs Raspberry Pi
				\item What is the major factor in the choice 
				\item How are the sensors wired to the processor 
				\item How to read the  data
				\item What is the effective resolution needed for each application
			\end{itemize}
			
		\end{enumerate}
	\subsection{Mesh network}
	A mesh network is a type of network that uses multiple devices to relay data between each other, making a decentralized network.
	The mesh to be used is a wireless mesh network which is created through the connection of wireless access point(WAP) nodes.
	Wireless mesh networks work through mesh nodes, mesh clients and gateways:
	\begin{enumerate}
		\item Mesh node

		nodes act as mesh routers and endpoints
		\item Mesh clients

		these are  end devices
		\item Gateways

		Data passes through the gateway as it enters or exits a network
	\end{enumerate}

	The following is a block diagram of a mesh network:
	\begin{figure}[h!]
	    \begin{center}			
	    \includegraphics[width=0.5\linewidth]{Images/basic mesh network diagram.png}\par
	    \caption{Basic block diagram of a mesh network}
	
	    \label{Basic block diagram of a mesh network}
	     \end{center}
	\end{figure}

	Each node will be attached to a tree, each having a transceiver

% Body of the literature review
\newpage
\section{Hardware Consideration}
In this project we need to have data to transmit firstly let's describe what we want our network to have:
\begin{enumerate}
	\item we want our mesh network to transmit  data for  example  temperature, humidity and  light and camera
	\item I want there to  be data read every hour  and  stored  as a CSV file  the image file  will depend on the module I pick
	\item I want to have a  motion to  detect  any animal that passes  the  node 
\end{enumerate}
\newpage
the following is  a  rough circuit diagram  for the project:
\begin{figure}[h!]
	\centering
	\includegraphics[width=0.5\linewidth]{Images/block_diagram_for_mesh_device.png}
	\caption{Rough circuit diagram for project}
	\label{Rough circuit diagram for project}
\end{figure}
firstly let's establish the following:
\begin{enumerate}
	\item I can't use  PCB due to  the ordering process taking too long to come due to the  time given  to me 
	\item using any  type of board like wire wrap  would take too long and is  outside of  the  goals of this project
	\item This leaves with  a  choice of either the  Arduino or pi
\end{enumerate}
This  section will Dicuss the following:
\begin{enumerate}
	\item The sensors we will use in the project
	\item the ADC we will have  to will have to  consider
	\item the camera i picked considation  for  in this  project
	\item the memory module condestrations
	\item the battery i  picked
	\item Considering the ardunio vs PI
\end{enumerate}

\subsection{Sensor considerations}
In this section, we will discuss  the process of  considering each commponet of the sensors
these sensor will be the following:
\begin{enumerate}
	\item Temperature
	\item Humdity
	\item Light
	\item Motion
\end{enumerate}
\subsubsection{Temperature \& Humidity sensor}
In our consideration for  this  sensor we can establish that we want our sensor to work in the following  conditions:
\begin{enumerate}
	\item our mesh node will be outside
	\item Our device is in Ireland		\item Our device is in a forest
\end{enumerate}
From that knowledge, I researched the  temperature range in Ireland,


According to  Met eireann\cite{Eirrean}, we get the following table which  the highest temperature in a  Shaded
\begin{table}[h!]
	\begin{tabular}{ | c | c | c | }
		\hline
		Highest Shaded Air (°C) & Station & Date \\ \hline
		18.5°C & Dublin (Glasnevin) & 10th 1998 \\ \hline
		18.1°C & Dublin (Phoenix Park) & 23rd 1891 \\ \hline
		23.6°C & Dublin (Trinity College) & 28th 1965 \\ \hline
		25.8°C & Donegal (Glenties) & 26th 1984 \\ \hline
		28.4°C & Kerry (Ardfert Liscahane) & 31st 1997 \\ \hline
		33.3°C & Kilkenny (Kilkenny Castle) & 26th 1887 \\ \hline
		33.0°C & Dublin (Phoenix Park) & 18th 2022 \\ \hline
		31.7°C & Carlow (Oak Park) & 12th 2022 \\ \hline
		29.1°C & Kildare (Clongowes Wood College) & 1st 1906 \\ \hline
		25.2°C & Kildare (Clongowes Wood College) & 3rd 1908 \\ \hline
		20.1°C & Kerry (Dooks) & 1st 2015 \\ \hline
		18.1°C & Dublin (Peamount) & 2nd 1948 \\ \hline
		\end{tabular}
		\caption{Highest shader air Met Eireann(13$^{th}$ June 2023)}
		\label{Highest shader air Met eirrean}
	\end{table}

According to the table, the highest temperature is 33.3  
now to look at the  other  extreme for the Lowest temperature:
	\begin{table}[h!]
		\begin{tabular}{ | c | c | c | }
		\hline
		Lowest Shaded Air (°C) & Station & Date \\ \hline
		-19.1°C & Sligo (Markree) & 16th 1881 \\ \hline
		-17.8°C & Longford (Mostrim) & 7th 1895 \\ \hline
		-17.2°C & Sligo (Markree) & 3rd 1947 \\ \hline
		-7.7°C & Sligo (Markree) & 15th 1892 \\ \hline
		-5.6°C & Donegal (Glenties) & 4th 1979 \\ \hline
		-3.3°C & Offaly (Clonsast) & 1st 1962 \\ \hline
		-0.3°C & Longford (Mostrim) & 8th 1889 \\ \hline
		-2.7°C & Wicklow (Rathdrum) & 30th 1964 \\ \hline
		-3.5°C & Offaly (Clonsast) & 8th 1972 \\ \hline
		-8.3°C & Sligo (Markree) & 31st 1926 \\ \hline
		-11.5°C & Wexford (Clonroche) & 29th 2010 \\ \hline
		-17.5°C & Mayo (Straide) & 25th 2010 \\ \hline
		\end{tabular}
	\caption{Lowest shader air Met Eireann(13$^{th}$ June 2023)}
	\label{Lowest shader air Met eirrean}	
	\end{table}

According to the table above the lowest temp is -19.1
In consideration for where the project our condition was a range of -19.1\textdegree C to 33.3\textdegree C.
\newpage
I also  looked  at  humdity this  referes to the amount of water vaper in the air. from  met eirrean \cite{eirrean2}
got this  table:
\begin{table}[h!]
	\begin{tabular}{|c|c|c|c|c|c|c|c|c|c|c|c|c|c|}
		\hline
		\space & Jan & Feb & Mar & Apr & May & Jun & Jul & Aug & Sep & Oct & Nov & Dec & Year \\
		\hline
		Mean at 0900UTC &87.0 &86.4&84.0&79.5&76.9&76.7&78.5&81.0&83.4&85.5&88.5&88.0&83.0 \\
		Mean at 1500UTC &80.6&75.7&71.0&68.3&68.0&68.3&69.0&69.3&71.5&75.1&80.3&83.1&73.3\\
		\hline
	\end{tabular}
	\caption{Realtive Humidity(\%) according to met eirrean}
	\label{Realtive Humidity according to met eirrean}
\end{table}
The ranges are 68.3\% to 88 \% So with these  conserdations here are  the  diffrent components:

\begin{table}[h!]
	\centering
	\includegraphics[width=0.5\linewidth]{Images/tempssenorscompared.png}
	\caption{Comparing of temperature sensors}
	\label{Comparing of temperature sensors}
\end{table}
After this, I limited this down to two sensors DHT22 and DHT11. The  following are the advantages and disadvantages of the DHT22 and DHT11:
\begin{table}[h!]
	\centering
	\scalebox{0.8}{\begin{tabular}{|c|c|c|}
	\hline
		Device & Advantages & Disadvantages  \\
		\hline
		\hline
		DHT22 & good accuracy has temp and humidity, falls in our temp range & sample period 2 seconds \\
		\hline
		DHT11 & OK voltage,better sample period & draws a lot of current , and our of range \\
	\hline
	\end{tabular}}
	\caption{Comparing DHT22 and DHT11}
	\label{Compareing DHT22 and DHT11}

\end{table}
So in conclusion I choose DHT22 which is a  Digital output. See a wiring diagram below
This will have an Interface of the following:

\begin{figure}[h!]
	\centering
	\begin{subfigure}{0.4\textwidth}
		\includegraphics[width=\textwidth]{Images/InterfaceforDHT22.png}
		\caption{Interface for DHT22}
		\label{Interface for DHT22}
	\end{subfigure}
	\hfill
	\begin{subfigure}{0.4\textwidth}
		\includegraphics[width=\textwidth]{Images/schematicforDHT22.png}
		\caption{Schematic for DHT22}
		\label{Sychematic for DHT22 revised}
	\end{subfigure}
\end{figure}
From above we see our schematic, DHT22 connections are the following:
\begin{itemize}
	\item VDD is connected to 5v of the pi
	\item the Data pin is connected to GPIO 3
	\item Gnd pin  of the  pi is  connected to the ground  of DHT22 

\end{itemize}
\cite{sparkfun} The following is the \href{https://www.sparkfun.com/datasheets/Sensors/Temperature/DHT22.pdf}{link} to the datasheet of this module
when reading from this  componet there is  a  delay  of 2 second due to the  sampling period.

\subsubsection{Light sensor}
In this section, we want to consider the following:
\begin{enumerate}
	\item What region are we in 
	\item What light levels do we  expect in this  country
	\item What sensor  will  accommodate this  range
\end{enumerate}
\newpage
For this sensor we also must consider the outside aspect of the  project  i found this table on \cite{wiki_2023}
	\begin{table}[h!]
	\centering
	\begin{tabular}{|l|l|}
	\hline
		Imminence & Example \\ \hline
		**0.002 lux** & Moonless clear night sky \\ \hline
		**0.2 lux** & Design minimum for emergency lighting (AS2293). \\ \hline
		**0.27 \& 1 lux** & Full moon on a clear night \\ \hline
		**3.4 lux** & Dark limit of civil twilight under a clear sky \\ \hline
		**50 lux** & Family living room \\ \hline
		**80 lux** & Hallway/toilet \\ \hline
		**100 lux** & Very dark overcast day \\ \hline
		**300 to 500 lux** & Sunrise or Sunset on a clear day. Well-lit office area. \\ \hline
		**1,000 lux** & Overcast day; typical TV studio lighting \\ \hline
		**10,000 to 25,000 lux** & Full daylight (not direct sun) \\ \hline
		**32,000 to 130,000 lux** & Direct sunlight \\ \hline
	\end{tabular}
	\caption{Illuminates values}
	\label{Illuminates values}
\end{table}
	This table is the  assoicated lux level  incate when the vaules are . 
	From  above we want our sensor to be 0.002 to 25000 lux ideally, with that in mind here are the components I found  through research:
	\begin{table}[h!]
	\small
	\centering
	\begin{tabular}{|l|l|l|l|l|}
	\hline
		Modules & Voltage Range & Analogue /Digital Outputs & illumination range & Current rating \\ 
		\hline
		LM393 with GL5528 & 3.3v to 5v & Analogue & 0 lux to 100lux & 250nA \\ 
		\hline
		DFR0026 & 3.3v to 5v & Analogue & 1 Lux to 6000 Lux & 120uA \\ \hline
		LM393 with n5ac501085 & max 150V & Analogue & 10 lux to 100lux & 1mW \\ 
		\hline
		LM393 with NSL-06S53 & max 100v & analogue & 1 to 100 & 50mw \\ \hline
	\end{tabular}
	\caption{table of light sensors}
	\label{table of light sensors}
\end{table}
\newpage
After doing research DFR0026 \cite{DFR0026} is the option I propose to use as it is the best for our application  which will have an analogue  output to see the interface see below:

\begin{figure}[h!]
	\centering
	\includegraphics[width=0.5\linewidth]{Images/InterfaceofDFR0026.png}
	\caption{Interface for  DFR0026}
	\label{Interface for  DFR0026}
\end{figure}

The following are  the connections:
\begin{enumerate}
	\item VCC pin is connected  to 5v
	\item Gnd of the  sensor is connected to Gnd of the Pi
	\item The output is connected to  ch 0
	\item the output ranges  from  0 to  5 v
\end{enumerate}
The commpoent relies on the  ADC  which  is on page\pageref{Adc section}
\subsubsection{Motion sensor}

For this section, we have  to consider the following:
\begin{enumerate}
	\item The range of the  sensor
	\item The degree of the  sensor
	\item How long of a  delay is the sensor
\end{enumerate}
The  following are  the components I considered:
\begin{table}[h!]
	\centering
	\begin{tabular}{|c|c|c|c|c|c|}
		\hline
		Modules & Voltage Range & Distance & Max angle & Analogue /Digital Outputs & Power \\
		\hline
		HC-SR501 & 5-20V & 3 to 7m & 110 & Digital & 50uA \\
		AM312 & 4.5-20v & 3m & 130 & Digital & 60uA \\
		AS312 & -0.3 - 3.6V & 12m & 130 & Digital & 100mA \\
		\hline
	\end{tabular}
	\caption{Motion sensor components}
	\label{Motion sensor components}
\end{table}

The sensor I'm choosing is AS312\cite{micros}(which has a delay time of 2 seconds) which is a Digital interface to see the wiring see below:
\newpage
the following is the interface for our device

\begin{figure}[h!]
	\begin{center}
		\includegraphics[width=0.5\linewidth]{Images/interfaceofAS312.png}
	\caption{Interface for AS312}
	\label{Interface for AS312}
	\end{center}

\end{figure}

The connections are the following:
\begin{enumerate}
	\item VCC is connected  to 5v pin of the Pi
	\item GND is connected to the GND of the  Pi
	\item Vout is connected to GPIO 27
\end{enumerate}
This  component has the following:
\begin{enumerate}
	\item Range  of 12 meters 
	\item An  angle  of  65$^o$ degree
	\item A Delay of 15 \mu Seconds
\end{enumerate}

\subsubsection{Radio Module}
For this section, we have the following considerations:
\begin{itemize}
	\item The devices are in a forest
	\item Meaning  Gigahertz  isn't  a desirable frequency
	\item We want a module that low low-power
	\item a model that  will have a high throughput 
\end{itemize}

Through research, I found the following table:
\begin{table}[h!]
	\centering
	\includegraphics[width=0.5\linewidth]{Images/radiomoudles.png}
		
	\caption{Radio modules found in research}
	\label{Radio modules found in research}
	
\end{table}

Out of these, I picked the MM2 Series 900 MHz\cite{freewave}.Note that the seller of this  radio module has  limited the documentation  of this module makes it hard to  draw an interface for this module which will be done  next  semester
\subsection{ADC Considerations}
\label{Adc section}
for the ADC  the following considerations:
\begin{enumerate}
	\item low power
	\item high bit resolution
	\item low amount of channels
	\item high sample rate
\end{enumerate}
the two things we  want  for this is the high bit  Resolution  and  a  high sample rate
\begin{table}[h!]
	\begin{center}
		\begin{tabular}{|c|c|c|c|c|}
			\hline
			Device & Resolution & Sample rate & Input range & Power consumption \\
			\hline
			ADC pi Zero & 17 bits & 100KHz & 0-5.06v & 10mA \\
			MCP3008 & 10 bits & 200 ksps & 2.7v- 5.5v & 500uA \\
			\hline
		\end{tabular}
	\end{center}
\end{table}

Above are the components I had to choose from 
for this project, I picked  MCP3008 due to its  resolution and  sample rate
the following is the   schematic for the  MCP3008\cite{ada}
\begin{figure}[h!]
	\centering
	\includegraphics[width=0.5\linewidth]{Images/SchematicforMCP300.png}
	\caption{Schematic for  MCP3008}
	\label{Schematic for  MCP3008}
\end{figure}
the following are connections:
\begin{enumerate}
	\item VDD is connected to 3v3 pin of the  Pi
	\item VRef is  also connected to  3v3 pin of the  Pi
	\item AGND is connected to the gnd pin 
	\item CLK pin is connected to GPIO port 11
	\item Dout pin is connected to GPIO port 9
	\item Din pin is connected to GPIO port 19
	\item CS pin  is connected  to GPIO port 8
	\item DGND ping is connected to the gnd pin
\end{enumerate}
this  commponet has the following:
\begin{enumerate}
	\item A 10 bit resultion
	\item seen as  the  reference  will  be 3.3v  
	\item 200ksps meaning the delay to read is 5$\mu$ seconds
\end{enumerate}
\subsection{Camera}
For the camera, we have to keep in mind the following:
\begin{enumerate}
	\item Focal length
	\item Resolution
	\item Power
	\item what lux values it works at
\end{enumerate}
\begin{table}[h!]
	\centering
	\scalebox{0.6}{\begin{tabular}{|c|c|c|c|c|c|c|c|c|}
	
		\hline
		Modules & Voltage range & lens size & Image Resolution & Video Resolution & Frame Rate & Type of Output & Preferred condition & Power \\
		Raspberry Pi VR 220 Camera & 3.3V ac &can change with lens & 3280 X 2462 & 1920 x 1080 &  30 FPS & Need to research & Daytime & 38mA \\
		DIGILENT 410-358 & 3.6v &  optical size 1/4 inches  & 2592 x 1944 &? &? &Digital &? & 200mA \\
		The Raspberry Pi NoIR & 3.3v  & 1/4 inches  &3280 x 2464  & 1080 or 720  &30 60 fps &need to research & house & 38mA \\
		OV7670 VGA & 2.45 to 3.0v ac & 1/6 inches & 2.36mm x 3.6um &? & 30 fps & analogue &  need to research &60mW \\
		\hline
	\end{tabular}}
	\caption{Camera module}
	\label{Camera module}
\end{table}
The camera I pick is a Raspberry Pi VR 220\cite{RS} Camera to see how to connect look at the  following \href{https://youtu.be/yhM1NhD-kGs?si=yxgFZb84yxSGLtM3}{link} 
\subsection{Memory module}
For this section, we consider the following:
\begin{enumerate}
	\item The file formatting of the sensor data
	\item The file formatting of  the camera data
	\item What are  the possible sizes of data
	\item what is the size of the  raspberry pi OS
\end{enumerate}
in my project, I plan on using the following:
\begin{enumerate}
	\item For the sensor I plan on use storing the data  in a  CSV file   with the following heading: (timestamp, heat, humidity, light level, anything detected) which can be around 25KB
	\item For the camera in the plan using 10 MB is the  largest file  size
	\item For the raspbery pi i downloaded  the raspberry imager  this has  loads of  options such as  the  following on \pageref{pi os}
\end{enumerate}
after has been we must consider a  mircoSD ,here is the following considerations:
\begin{table}
	\begin{tabular}{|c|c|c|c|c|c|c|c|c|c|c|c|}
		\hline \\
		product name & Capacity & 
		\hline
	\end{tabular}
	\caption{mirco SDs in consirdation}
	\label{mirco SDs in consirdation}
\end{table}
this depends on the onboard storage but here is what I found through research:
\begin{table}[h!]
	\centering
	\includegraphics[width=0.8\linewidth]{Images/memory_devices.png}
	\caption{Memory usb to consider}
	\label{Memory usb to consider}
\end{table}
The Raspberry Pi 4 supports USB 2 and  USB 3. For this, I'll pick the Turbo 1GB USB 2 Flash Drive

\subsection{Battery}
In this section, we want to consider the  following:
\begin{enumerate}
	\item Have enough power for all  sensors  and  radio module
	\item Have storage of the battery
	\item Discharge rate of the  battery (how many operating hours can I get out of the  battery)
\end{enumerate}
Here are the following  Devices I found :
\begin{table}[h!]
	\centering
	\begin{tabular}{|c|c|c|c|c|c|}
		\hline
		Modules & Voltage & Interface & Power & Chemistry & Supply time\\
		\hline
			Li-polymer Battery HAT  & 5v & Micro USB & 1.8A &lithium battery &5 hours \\ \hline
	\end{tabular}
	\caption{battery considerations}
	\label{battery considerations}
\end{table}

The battery I'm going for is the li-polymer which has a micro USB 
how to charge:
\begin{itemize}
	\item Step 1: Insert the Li-polymer battery into a 2.0mm battery socket
	\item Step 2: Connect the power adapter to a micro USB or Type-C interface by USB cable.
\end{itemize}
Aside: this commpoent has the following:
\begin{enumerate}
	\item A battery that is 3.7v 3000$mAh$ 
	\item Output voltage of 5 volts
	\item an estimated Power supply time  of  5 hours
\end{enumerate}
\subsection{Arduino vs PI Consideration }
In this project, we will have to choose between what microprocessor we will use. we can have 3 options
\begin{enumerate}
	\item PCB (printed circuit board)
	where we design the circuit in a program like Fusion 360. The major issue is due to  the current state of  silicon chips which will slow down  the progress of 
	the implementation stage
	\item Arduino 
	\item Raspberry Pi
\end{enumerate}
for these will consider the Arduino and  the Raspberry Pi	the Advantages and  disadvantages of these are the following:
\begin{table}[h!]
	\centering
	\begin{tabular}{|c|c|}
		\hline
		Arduino & pi \\
	
	\hline \hline
	Advantages & Advantages \\
	\hline \hline
	1. Arduino has a 10-bit ADC & 1. Pi can compile Python (easier to write ) \\

	\hline \hline
	Disadvantages & Disadvantages \\
	\hline \hline
	1. Arduino has a supper set of C++ & 1. Pi is a technically a small CPU \\
	2. Arduino only has 6 Analogue pins  & 2. The pi needs an ADC circuit to deal with inputs that are analogue \\
	
	\hline
	\end{tabular}
	\caption{Advantages /Disadvantages of Arduino vs pi}
	\label{Advantages /Disadvantages of Arduino vs pi}
\end{table}

Although the  Arduino would be more efficient than the Raspberry Pi due to
Raspberry Pi has an Operating System I am picking the Pi as I'm more familiar with Python and Linux. Linux can  be used to handle the networking side  of   the project
I am willing to lose some efficiency in power for an easier time making the code for  this  project 
\subsubsection{Picking a Raspberry Pi}
Now that we have  picked out a device to  use  we need to define what we need in terms of   the following:
\begin{enumerate}
	\item The amount of GIPO PORTS we need 
	\item Nature of the output of the sensor
	\item Speed of the clock
\end{enumerate}	
GPIO(General purpose input/output) is used  to select the input/output the pi can only take in  digital signals only
Seen as we have our components chosen that require A GPIO port (temperature/ humidity, on page \pageref{Compareing DHT22 and DHT11}, Light on page \pageref{table of light sensors}, motion on page \pageref{Motion sensor components})
we need at least  3 GPIO ports to be available to us as the light sensor and the motion will  need  an adc as looking through the documentation .firstly let's look at the  different models
\begin{table}[h!]
\scalebox{0.6}{\begin{tabular}{|l|l|r|l|l|l|1|}
\hline
\rowcolor[HTML]{CE6301} 
Raspberry Pi Model      & Internal Clock Speed & \multicolumn{1}{l|}{\cellcolor[HTML]{CE6301}Power (Watts)} & GPIO Features & Type of Connectors                                                            & SRAM                   \\ \hline
Raspberry Pi 1 Model B+ & 700 MHz              & 5.5                                                        & 26 GPIO pins  & 1 HDMI, 1 micro USB, 1 USB 2.0, 1 audio jack                                   & 512 MB                 \\
Raspberry Pi 2 Model B  & 900 MHz              & 7.5                                                        & 40 GPIO pins  & 1 HDMI, 1 micro USB, 4 USB 2.0, 1 audio jack                                   & 1 GB                   \\
Raspberry Pi 3 Model B+ & 1.4 GHz              & 8                                                          & 40 GPIO pins  & 1 HDMI, 1 micro USB, 4 USB 2.0, 1 audio jack, 1 Gigabit Ethernet, 1 PoE header & 1 GB                   \\
Raspberry Pi 3 Model A+ & 1.4 GHz              & 5                                                          & 26 GPIO pins  & 1 HDMI, 1 micro USB, 2 USB 2.0, 1 audio jack                                   & 512 MB                 \\
Raspberry Pi Zero       & 1 GHz                & 1.2                                                        & 40 GPIO pins  & 1 mini HDMI, 1 micro USB, 1 micro-USB OTG                                      & 512 MB                 \\
Raspberry Pi Zero W     & 1 GHz                & 1.3                                                        & 40 GPIO pins  & 1 mini HDMI, 1 micro USB, 1 micro-USB OTG, 1 Wi-Fi/Bluetooth module            & 512 MB                 \\
Raspberry Pi Zero 2 W   & 1 GHz                & 0.8                                                        & 40 GPIO pins  & 1 mini HDMI, 1 micro USB, 1 micro-USB OTG, 1 Wi-Fi/Bluetooth module            & 512 MB                 \\
Raspberry Pi 4 Model B  & 1.5 GHz              & 7                                                          & 40 GPIO pins  & 2 HDMI, 2 USB 3.0, 2 USB 2.0, 1 Gigabit Ethernet, 1 audio jack                & 1 GB, 2 GB, 4 GB, 8 GB \\ \hline
\hline
\end{tabular}}
\caption{Table of Raspberry Pi's}
\label{Table of Raspberry Pi's}
\end{table}

The above table displays the modules seen as our radio modules is 900Mhz  we want  1.5GHZ which is the Raspberry Pi 4  which needs USB\- c charger and an HDMI.
\subsubsection{Picking an PI OS}
\label{pi os}
Seen as 
\subsection{Conclusion}
In this project the hardware needed is the  following: these are from 
\begin{enumerate}
	\item 1 x Raspberry Pi 4 Model B 
	\item 1 x HDMI cable
	\item 1 x USB\-C cable
	\item 1 x USB \-C charging head
	\item 1 x DHT22
	\item 1 x DFR0026
	\item 1 x AS312
	\item 1 x MM2 Series 900 MHz
	\item 1 x MCP3008
	\item 1 x Raspberry Pi VR 220 Camera
	\item  1 x Li-polymer Battery HAT 
	\item 1 x Turbo 1GB 
	
	
\end{enumerate}

\newpage
\section{Software considerations}
Having established the essential hardware needed for this project. Next, consider the following  for the software of the  project:
	\begin{enumerate}
	    \item How to structure code 
	    \item Linux set up of sever and nodes
	    \item How will data be sent
	    \item Will this be an OOP or functional approach?
	    \item How to program each device?
	\end{enumerate}
	\subsection{Raspberry Pi OS}
	\label{pi os}
	In this section, it must be kept in mind  that each OS is  heavyweight the following needs to be considered:
	\begin{enumerate}
		\item If the SD is formatted the data on the SD is lost. Does it corrupt the card?
		\item An os that is low in capacity 
		\item Is a desktop needed or can we use the terminal?
		\item How does the OS  respond to USB drives?
	\end{enumerate}

	According to the \cite{projects} the imager will erase all the data while installing the os. From research, the suggestion of backing up the data is a good suggestion.
	for now, the recommended OS is used. and strip down as the project progresses. which will be discussed in the methodology section of this report.
	
	\subsection{Sensor code}
	In this section, the following will be discussed :
	\begin{enumerate}
		\item DHT22
		\item AS312
		\item MCP3008
		\item DFR0026
		\item Kuman for Raspberry Pi 3B+ TFT LCD Display
		\item Raspberry Pi VR 220 Camera 
	\end{enumerate}

	the project code will mainly be object-oriented. so the goal is to first test it with my laptop and  Create A bash file full of commands to install the libraries, making the code split up into different parts so that all that is needed is the libraries used and code that won't all have to be compiled in one file.

	\subsubsection{DHT22}
	In this section we have to consider the following: 
	\begin{enumerate}
		\item The GPIO port as on page \pageref{Sychematic for DHT22 revised} This is connected to port 3 
		\item The type of output is digital so no  extra hardware/code is needed
	\end{enumerate}

	The following is a rough guide on how to read from the DHT22 from the following \href{https://www.instructables.com/Raspberry-Pi-Tutorial-How-to-Use-the-DHT-22/}{link}.
	Firstly open the terminal in the Pi and
	type the following commands:
	\begin{lstlisting}[style=bashstyle]
		git clone https://github.com/adafruit/Adafruit_Python_DHT.git
		cd Adafruit_Python_DHT
		sudo apt-get update
		sudo apt-get install build-essential python-dev
		sudo python setup.py install
	\end{lstlisting}
	the code does the following:
	\begin{enumerate}
		\item firstly git clone will clone the  repository onto to device
		\item Then change directories  a
		\item update Linux
		\item install dev kit for  python 
		\item and install the setup 
	\end{enumerate}
	
	this will then lead to  the  following code:
	\begin{lstlisting}[style=mystyle,caption={Example code for DHT2},numbers=left,firstnumber=1]
		#Libraries
		import Adafruit_DHT as DHT
		from time import sleep
		def setup_DHT22(Gpoiport:int):
		humidity,temp=dht.read_retry(DHT.DHT22, Gpoiport)
			sleep(5)
			return humidity, temp
		h,t=setup_DHT22(3)
		print('Temp={0:0.1f}*C  Humidity={1:0.1f}%'.format(t,h))
	\end{lstlisting}
	this code will do the following:
	\begin{enumerate}
		\item Import DHT from the Adafuit library
		\item in the  function which takes the GPIO port  as an integer this will read the data on the pin and  print it  out
	\end{enumerate}
	
	\subsubsection{AS312}
	for this section i followed  this \href{https://pimylifeup.com/raspberry-pi-motion-sensor/}{link}
	we also want to keep in mind the following:
	\begin{enumerate}
		\item This has a  digital interface and is connected  to GPIO 27 
	\end{enumerate}
	Here are the rough steps firstly  type the following into the  terminal 
	\begin{verbatim}
		sudo apt-get install python-rpi.gpio
	\end{verbatim}
	which will install a gpio python module
	Then type this into an IDE of your  choosing
	\begin{lstlisting}[style=mystyle,caption={Example code for AS312},numbers=left,firstnumber=1]
		import RPi.GPIO as GPIO
		import time

		pir_sensor = 27
		GPIO.setmode(GPIO.BOARD)

		GPIO.setup(pir_sensor, GPIO.IN)
		current_state = 0
		
		time.sleep(0.1)
		current_state = GPIO.input(pir_sensor)
		if current_state == 1:
			print("GPIO pin %s is %s" % (pir_sensor, current_state))
			# trigger camera
		# must look up this 
		GPIO.cleanup()
	\end{lstlisting}
	this code does the  following:
	\begin{enumerate}
		\item it will look at the  pin for a pulse 
		\item Once it senses a pulse  it will trigger  the camera
	\end{enumerate}
	\subsubsection{DFR0026}	
	from the last example, nothing has changed from the last component
	an example code for this can be found on page \pageref{adc code}
	\subsection{MCP3008}
	for this section, we want to consider the following:
	\begin{enumerate}
		\item The MCP3008 data out is GIPO 9 
	\end{enumerate}
	This section follows this \href{https://randomnerdtutorials.com/raspberry-pi-analog-inputs-python-mcp3008/}{link}
	firstly try the following in command in the terminal 
	\begin{verbatim}
	sudo raspi-config nonint do_spi 0
	\end{verbatim}
	
	\label{adc code}
	\begin{lstlisting}[style=mystyle,caption={ADC code},numbers=left,firstnumber=1]
		from gpiozero import MCP3008
		from time import sleep
		DFR0026 = MCP3008(channel=0, device=0,port=9)
	
		print ('raw: {:.5f}'.format(DFR0026.value))
		sleep(0.1)
	\end{lstlisting}
	this code will select a  channel and device, port and  print the values of the ADC's
	\subsection{Raspberry Pi VR 220 Camera}
	to get started with this simply look at the following \href{https://projects.raspberrypi.org/en/projects/getting-started-with-picamera/4}{link}
	here is an example of the code of this module :
	\begin{lstlisting}[style=mystyle,caption={example code for camera},numbers=left,firstnumber=1]
		from picamera import PiCamera
		from time import sleep

		camera = PiCamera()

		camera.start_preview()
		sleep(5)
		camera.stop_preview()
	\end{lstlisting}
	this will  take a photo of what is in front of the  camera

	\subsection{MM2 Series 900 MHz}
	for this section, the seller of this module has no public  documentation so it is hard to  come up with  an make interface for  this section   
	\subsection{code structure}
	The code structure for this  will be an object-oriented program all the individual sensors and  hardware  for the pi will be as displayed above the code in this section will be formatted into objects for example I will have an  object   called proj\_sensor and  a method of this  would be  DHT22 while an attribute of this would  be  the  sample rate
	the following is a  rough breakdown of the  structure of the code
	\begin{itemize}
		\item Sensor object
		
		\begin{itemize}
			\item Temperature and humanity method
			\item light method
			\item Motion method which triggers the camera
			\item Battery method which is a constructor method
			\item Memory method which  links with the radio 
		
		\end{itemize}
		
		\item radio object which reads from Memory and  transmits the data 
	
	\end{itemize}
	\subsection{File structure}
	For the  File structure, we want our sensor data to be stored every hour in a  CSV file with the following column headings:
	\begin{enumerate}
		\item timestamp
		\item Heat
		\item Humidity
		\item light level
		\item motion detected (True/False)
	\end{enumerate}
	for the writing to Date, we will use Pandas to write to the CSV file
	for file sorting, I will use the Python Library glob  which I can use  to look for  files 
	the following is an example of how  to  make a CSV file:
	firstly let's make a data frame:
	\begin{lstlisting}[style=mystyle,caption={sample code for turning sensor data into a data},numbers=left,firstnumber=1]
		import pandas as PD
		import numpy as np
		from datetime import datetime
		cols_name=["Timestamp", "Temperature", "Hummidty", "Light_level", "Motion_dected"]

		#assume that being recorded now
		data=[]
		timestap=datetime.now()
		timestap=timestap.strftime("%d/%m/%Y %H:%M:%S")
		Current_state=1
		Heat=0.40
		Hummidty=1.0
		Light_level=0.23
		data=np.array([[timestap],[Heat],[Hummidty],[Light_level],[Current_state]])
		data=data.T
		df= pd.DataFrame(data,columns=cols_name)
	\end{lstlisting}
	Next, use the.To\_csv method from  pandas
	another Libraries that could  be useful is the Tkinter
	here is a  sample of how to   store where the  file is  gonna be:
	\begin{lstlisting}[style=mystyle,caption={example code for storing directory},numbers=left,firstnumber=1]
		import tkinter as tk
		from Tkinter import filedialog
		import json
		import os

		root = tk.Tk()
		root.withdraw()
		selected_dir = filedialog.askdirectory()

		if not os. path.exists('selected_dir.json'):
			# Write the selected directory to a JSON file
			with open('selected_dir.json', 'w') as f:
				json.dump(selected_dir, f)
				print("Successfully saved selected directory to JSON file.")
		else:
			print("File 'selected_dir.json' already exists. Not saving the directory.")

		root.quit()
	\end{lstlisting}
	Other useful Libraries allow you to  select all  .csv, png  called glob
	for our TDD Section, we will have to use the  following command:
	\label{TDD sample bash}
	\newpage
	\begin{verbatim}
		# !/bin/bash

		dir_name=$1

		size=$(du -sh "$dir_name" | cut -f1)

		echo "Directory size: $size"
	\end{verbatim}
	This is a script that will look at a  director this can be a home directory that will call the  space 13K
	the "| cut -f1" will only focus on the size string message and then print out the size. this is  just  a sample  script 
	\subsection{Test Driven development}
	In this  project ill will be using  Test Driven Development (TDD) is a software development approach where tests are written before the actual code
	the following are the advantages of TDD:
	\begin{enumerate}
		\item Advantages
		\begin{enumerate}
			\item TDD forces you to consider potential failure points and edge cases upfront, leading to earlier detection and resolution of bugs.
			\item TDD encourages you to think about the desired behavior and interfaces of your code
			\item TDD provides immediate feedback on whether your code works as intended,
		\end{enumerate}
	\end{enumerate}

\section{Attenuation}

Attenuation refers to a reduction in the strength of a signal.
 Attenuation occurs with any signal, whether digital or analogue. 
Seen the aim of making a network the first step is to look into  
what frequencies can be transmitted and received.
\newline
In the environment in which we want our project to take place, we want the following:
 \begin{enumerate}
	\item An antenna that a high so we can affect the data rate of the signal
	\item A frequency range at which Attenuation is not present 
 \end{enumerate}
Through research, I found the following plots:

\newpage

\begin{enumerate}

	\item  First Plot
	The first plot  I got for Savage e.t al pg. 7 \cite{Savage}
	\begin{figure}[h!]
		\centering
		\includegraphics[width=0.5\linewidth]{Images/Silver_Maple.png}
		\caption{Silver Maple in-leaf excess attenuation for the line of trees geometry (receiver antenna height: 3.5 m, SAVAGE ET AL.pg.7}
		\label{Silver Maple in-leaf excess attenuation for the line of trees geometry (receiver antenna height: 3.5 m, SAVAGE ET AL.pg.7}
	\end{figure}
 
This graph displays as vegetation depth increases  Attenuation rises. The problem with this graph is that it doesn't give an in-depth view of which attenuation occurs.
This then led me to look up the International Telecommunication Union  \cite{ITU} recommendations for Attenuation in wooded areas



	\item Second Plot
\begin{figure}[h!]
		\centering
		\includegraphics[width=0.5\linewidth]{Images/ITU attteuntion.png}
		\caption{Specific attenuation due to woodland (Recommendation ITU-R P.833-7 (02/2012) Attenuation in vegetation pg.5}
		\label{Specific attenuation due to woodland (Recommendation ITU-R P.833-7 (02/2012) Attenuation in vegetation pg.5}
		\end{figure}
	V is the vertical polarization
	H is the horizontal polarization

	From this graph we  can assume the following:
	\begin{enumerate}
		\item From a frequency $\ge$15GHz we can assume Attenuation is more components
		\item Around the 1 GHz range we  get low values of Attenuation
		\item in the  MHz range we get the best response
	\end{enumerate}
	from this, I selected the  range which is  $10^6 hz$
	
\end{enumerate}
\newpage
so now that  we  established our range let us consider  what happens when it rains

\cite{Sabetahd_Mousavi_Ghasemi_Vafaei_Poursorkhabi_Mohammadzadeh_Zandi_2022}

\begin{figure}[h!]
	\begin{center}
		
	\includegraphics[width=\linewidth]{Images/atteuntion_2.png}\par
	\caption{Predicted attenuation due to rain for the region, which is measured by using the ITU standards,(Source: Hindawi(2014))}
	\label{Predicted attenuation due to rain for the region, which is measured by using the ITU standards,(source: Hindawi(2014))}
		\end{center}
\end{figure}
Ideally, we want a low MHz but we want speed and this  is dictated by what we choose let's further see how radio waves are affected by water/rain
\subsection{Absorption of water}
for this, I found this graph from Lunken Heimer \cite{lunken} 

\begin{figure}[h!]
	\centering
	\includegraphics[width=0.5\linewidth]{Images/absorstion.png}
	\caption{absorption of water}
	\label{absorption of water}
\end{figure}

According to the  graph, Water absorbs MHz frequencies which will affect the  transmission 
in the transmission and in some cases, we might have to consider non-line-of-sight communication when it rains or we might also consider another node to route to receive the node.





\section{mesh network considerations}
For this section, we have to consider the following:
	\begin{enumerate}
		\item How are we setting up the network
		\item What framework are we using to set this Up
		\item What are the advantages/disadvantages
	\end{enumerate}
	In my research I found two main frameworks that this project could use to achieve the mesh network these are the following:
	\begin{enumerate}
		\item LORA
		\item Zigbee
	\end{enumerate}
	According to Chen (2023)\cite{chen} "LoRa, as one of Low Power Wide Area Networks (LP-WANs) technologies, aims to enable IoT devices to perform long-range communications with lower power consumption [18]. LoRa makes use of the chirp spread spectrum (CSS) modulation to improve the transmission distance up to kilometres and also be resistant to multi-path effects."
	\par
	According to Vlad\cite{Gavra_Pop_Dobra_2023}, "ZigBee is an LP-WPAN (Low-Power-Wireless Personal Area Network) with short range and low power consumption, as mentioned before. The range for ZigBee devices is up to fifty meters and it is characterized by a low data rate, having a maximum value of 250 kbps. The protocol is suitable for sensors and IoT applications because of the low data rate and low power consumption"
	\par
	the following are the  differences between the two:
	\begin{table}[h]
		\centering
		
		\begin{tabular}{|c|c|c|c|}
		\hline
		\multicolumn{2}{|c|}{LoRa} & \multicolumn{2}{|c|}{ZigBee} \\
		\hline
		Advantages & Disadvantages & Advantages & Disadvantages \\
		\hline
		Long transmission distance & Low transmission rate & Low power consumption & Low data rate \\
		Low power consumption & Slow data transfer rate & Long range & Limited range \\
		Multi-channel information procession & Small payload & Scalability & Signal interference \\
		Strong anti-interface ability & Low bandwidth & $-$ & High-sensitivity levels \\
		High-sensitivity levels & Spectrum interference \\
		\hline
		\end{tabular}
		\caption{Advantages and Disadvantages of LoRa and ZigBee}
		\label{tab:lora_zigbee}
		\end{table}
		from research, these are very  similar  but it seems if I plan on adding lots of Zigbee is the best for this challenge  

\section{Review  key of research Papers}
The following are the research papers I used
\begin{enumerate}
	
	\item zhao
	
		In my research, I found  multiple projects that are similar to mine 
		In Zhao(2023)\cite[zhao]{zhao}  
		used  LORA to  track  light sensitivity, air pressure
		one of the challenges Zhao came across was Attenuation as stated above and also the author came across the problem of not having sufficient solar panels 
	\item Daniel
	
		Another paper I found in my research is by Daniel \cite{Daniel}
		In this, Daniel discusses modeling radio wave propagation in a forest environment which isn't in the scope of the project
		Daniel's work shows that a better approximation  for transmission loss was a key read to  under what happens on a more in-depth scale in my project
	
	\item Anna
	
		\cite{Anna} in Anna's paper  she  mainly used LORA where she compared line of sight and  the  non-line  line of sight environments in  urban and  forested areas
		this paper aims to study the effects of signal propagation in different environments.
	

	\item ITU
	
		\cite{ITU} in ITU in most research papers I found  it  referred back to this  document this document was  very helpful in terms  of  understanding Attenuation  and challenges that face

\end{enumerate}


\section{Summary}
This report  highlights the  challenges at come from  transmitting data in a  wooded area these challenges are  the following:
\begin{enumerate}
\item Attenuation 
\item Absorption
\end{enumerate}
In a wooded area, we established that  Attenuation occurs due to the reflection, and penetration of radio through any type of medium.
We established that our antenna will have to  be in the  Mhz range   but will still have signal loss /errors due to Absorption of  the  signal  received due  to rain or water being in the signal path
we have yet to consider the non-line of sight environment but this  is  to be  discussed when prototyping, this report mainly focuses on the hardware where the  focus is on  sensors such as:
\begin{itemize}
\item Temperature
\item Light
\item Motion
\item Humidity
\end{itemize} 
The report focuses on how to read this data from a  Software perspective the code will be an object-oriented program
where the code will be separated into different blocks of code so the file size is minimized and leads to a faster compile time. 